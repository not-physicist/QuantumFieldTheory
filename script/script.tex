\documentclass[oneside]{scrbook}
\KOMAoptions{fontsize=11pt, paper=a4}     
\KOMAoptions{DIV=13}                      

\usepackage[utf8]{inputenc}               
\usepackage[T1]{fontenc}                  
\usepackage[autostyle=true]{csquotes}     
\usepackage[varg]{txfonts}  			  %	Times-like fonts in support of mathematics
\usepackage{siunitx}   	  				  
\usepackage{enumitem}				      %	extra enumerate options

\renewcommand{\familydefault}{\rmdefault} % font to sans serif

%import external graphics and where to find these
\usepackage{graphicx}					  
\graphicspath{{figs/}}

\RequirePackage[backend=biber, style=numeric]{biblatex}
\addbibresource{refs.bib}

\usepackage{hyperref}
\RequirePackage[all]{hypcap}

%There are a number of symbols defined inside txfonts that are also defined in amsmath
% so you can just make these available again
\let\iint\relax
\let\iiint\relax
\let\iiiint\relax
\let\idotsint\relax
\usepackage{amsmath}
\usepackage{physics}
\usepackage{mathtools}
\usepackage{braket}
\usepackage{slashed} % feynman slash notation
\usepackage{simplewick} % wick contraction
\usepackage{tikz-feynman} % feynman diagrams 

% This will set fancy headings to the top of the page. The page number will be
% accompanied by the total number of pages. That way, you will know if any page
% is missing.
% If you do not want this for your document, you can just use
% ``\pagestyle{plain}``.
\usepackage{scrpage2}
\usepackage{lastpage}
\pagestyle{scrheadings}
\automark{section}
%\cfoot{\footnotesize{Page \thepage\ / \pageref*{LastPage}}}
\chead{}
\ihead{}
\ohead{\rightmark}
\setheadsepline{.4pt}

%restart footnotes every page
\usepackage{perpage}
\MakePerPage{footnote}
%symbols for footnotes
\usepackage[symbol]{footmisc}
\renewcommand{\thefootnote}{\fnsymbol{footnote}} % footnote mark with special symbols

% toprule and etc.
\usepackage{booktabs}

%%%%%%%%%%%%%%%%%%%%%%%%%% NEW COMMAND SECTION %%%%%%%%%%%%%%%%%%%%

%define equal
\newcommand{\defeq}{\vcentcolon =} 
\newcommand{\eqdef}{= \vcentcolon}
\newcommand{\euler}{\mathrm{e}}

%Lagrange density
\newcommand{\lag}{\mathcal{L}} 
%Hamiltonian density
\newcommand{\ham}{\mathcal{H}}

%identity matrix
\usepackage{dsfont}
\newcommand{\id}{\mathds{1}}

\newcommand{\vecnab}{\pmb{\nabla}}
\newcommand{\vecx}{\pmb{x}}
\newcommand{\vecy}{\pmb{y}}
\newcommand{\veck}{\pmb{k}}
\newcommand{\N}{\mathbb{N}}
\newcommand{\R}{\mathbb{R}}
\newcommand{\M}{\mathcal{M}}

%%%%%%%%%%%%%%%%%%%%%%%%%%%%% SETTINGS %%%%%%%%%%%%%%%%%%%%%%%%%%%%%%%%%%%%

\numberwithin{equation}{section}

%%%%%%%%%%%%%%%%%%%%%%%%%%%%%%%%%%%%%%%%%%%%%%%%%%%%%%%%%%%%%%%%%%

\title{Quantum Field Theory}
\author{Chenhuan Wang}
\date{\today}
\begin{document}
\maketitle
\tableofcontents

\chapter{Classical field theory}
\section{Field theory in continuum}
\paragraph{Euler-Lagrange-equation}
\begin{align}
	\partial_\mu \left( \frac{\partial \lag}{\partial(\partial_\mu \phi)} \right) - \frac{\partial \mathcal{L}}{\partial \phi} = 0
\end{align}
\paragraph{momentum density}
\begin{align}
	\pi(x) = \frac{\partial \lag}{\partial \dot{\phi}(\pmb{x})}
\end{align}
\paragraph{Hamiltonian density}
\begin{align}
	\ham(\phi(\pmb{x}),\pi(\pmb{x})) = \pi(\pmb{x}) \dot{\phi}(\pmb{x}) - \lag(\phi,\partial_\mu \phi)
\end{align}
\section{Noether Theorem}
If a Lagrangian field theory has an infinitisimal symmetry, then there is an associated current $j^\mu$, which is conserved.
\begin{align}
	\partial_\mu j^\mu &= 0 \\
	j^\mu &= \frac{\partial \lag}{\partial (\partial_\mu \phi)} \Delta \phi - X^mu
\end{align}
\paragraph{Energy-momentum tensor (stress-energy tensor)} \hspace{0pt}\\
Asymmetric version
\begin{align}
	\Theta^\mu_\nu = \frac{\partial \lag}{\partial(\partial_\mu \phi)} \partial_\nu \phi - \delta^\mu_\nu \lag
\end{align}
General version
\begin{align}
	T^{\mu\nu} = \Theta^{\mu\nu} + \partial_\lambda f^{\mu\nu\lambda}
\end{align}
with $f^{\lambda\mu\nu}=-f^{\mu\lambda\nu}$ or $\partial_\mu \partial\nu f^{\lambda\mu\nu}=0$


\chapter{Klein-Gordon theory}
\paragraph{(Real) Lagrangian density}
\begin{align}
	\lag = \frac{1}{2} \partial_\mu \phi \partial^\mu \phi - \frac{m^2}{2} \phi^2
\end{align}
\paragraph{Quantization}
\begin{align}
	\begin{split}
		\left[ \phi(\pmb{x}), \phi(\pmb{x}') \right] = \left[ \pi(\pmb{x}), \pi(\pmb{x}') \right] = 0 \\
		\left[ \phi(\pmb{x}), \pi(\pmb{x}') \right] = i \delta^{(3)}(\pmb{x} - \pmb{x}')
	\end{split}
\end{align}
\paragraph{Decomposition into Fourier modes}
\begin{align}
	\phi({\pmb{x}}) &= \int \frac{\dd^3 p}{(2\pi)^3}\frac{1}{\sqrt{2E_p}} \left( a_{\pmb{p}} e^{i\pmb{p}\cdot\pmb{x}} + a_{\pmb{p}}^\dagger e^{-i \pmb{p} \cdot \pmb{x}} \right) \\
	\pi({\pmb{x}}) &= \int \frac{\dd^3 p}{(2\pi)^3}(-i)\sqrt{\frac{E_p}{2}} \left( a_{\pmb{p}} e^{i\pmb{p}\cdot\pmb{x}} - a_{\pmb{p}}^\dagger e^{-i \pmb{p} \cdot \pmb{x}} \right)
\end{align}
thus the commutation relations for ladder operators:
\begin{align}
	\left[ a_{\pmb{p}}, a_{\pmb{p}'} \right] &= \left[ a^\dagger_{\pmb{p}}, a^\dagger_{\pmb{p}'} \right] = 0 \\
	\left[ a_{\pmb{p}}, a^\dagger_{\pmb{p}'} \right] &= (2\pi)^3 \delta^{(3)}(\pmb{p}-\pmb{p}')
\end{align}
\paragraph{Hamiltonian in terms of ladder operator}
\begin{align}
	H = \int \frac{\dd^3 p}{(2\pi)^3} E_p \left( a_{\pmb{p}} a_{\pmb{p}} ^\dagger + \frac{1}{2} \left[ a_{\pmb{p}}, a_{\pmb{p}}^\dagger \right] \right)
\end{align}
\paragraph{Normlisation} it's also lorentz-invariante
\begin{align}
	\braket{p|q} = 2E_p (2\pi)^3 \delta^{(3)}(\pmb{p} - \pmb{q})
\end{align}
\section{Heisenberg-picture fields}
\paragraph{Heisenberg-picture}
\begin{align}
	\ket{\psi_H} &= e^{iHt}\ket{\psi_s (t)} \\
	O_H(t) & = e^{iHt}O_S e^{-iHt} 
\end{align}
\paragraph{Field operator}
\begin{align}
	\phi(x) = \phi({\pmb{x},t}) &= \int \frac{\dd^3 p}{(2\pi)^3}\frac{1}{\sqrt{2E_p}} \left( a_p e^{ipx} + a_{p}^\dagger e^{-i p x} \right)
\end{align}

\section{Commutations and propogators}
\paragraph{Commutations}
\begin{align}
	\left[ \phi(x), \phi(y) \right] &= D(x-y) - D(y-x)
	\begin{cases}
		= 0 & \text{if $(x-y)$ is space-like} \\
		\neq 0 & \text{otherwise}
	\end{cases} \\
	D(x-y) &= \int\frac{\dd^3p}{(2\pi)^3} \frac{1}{2E_p} e^{-ip(x-y)}
\end{align}
\paragraph{Propogator}
\begin{align}
	\braket{0 | \phi(x) \phi(y) | 0} = D(x-y)
\end{align}
\paragraph{Feynman propagator}
\begin{align}
	\begin{split}
	D_F(x-y) &= \braket{0 | T \phi(x) \phi(y) | 0} \\
			 &= \Theta(x^0 - y^0)D(x-y) + \Theta(y^0 - x^0) D(y-x)
	\end{split}\\
	D_F(x-y) &= \int \frac{\dd^4 p}{(2\pi)^4}\frac{i}{p^2-m^2+i\epsilon} e^{-ip(x-y)}
\end{align}


\chapter{Quantization of the Dirac field}
\section{Dirac equation}
\begin{align}
	\left(i \gamma^\mu \partial_\mu - m \right) \phi(x) = 0
\end{align}
\paragraph{Standard representation (Dirac's)}
\begin{align}
	\gamma_0 = \begin{pmatrix} \id_2 & 0 \\ 0 & -\id_2 \end{pmatrix}
	\quad
	\pmb{\gamma} = \begin{pmatrix} 0 & \pmb{\sigma} \\ -\pmb{\sigma} & 0 \end{pmatrix}
\end{align}
\paragraph{Lorentz transformation}
\begin{align}
	\Lambda = \exp(\frac{1}{2}\omega_{\mu\nu}M^{\mu\nu})
\end{align}
with $\omega$ set of parameters and $M$ the generator of Lie algebra.
\paragraph{Spinor representation}
\begin{align}
	S^{\rho\sigma} &= \frac{1}{4} \left[ \gamma^\rho, \gamma^\sigma \right] = \frac{1}{2i}\sigma^{\rho\sigma} \\
\end{align}
\paragraph{Spinor transformation}
\begin{align}
	S(\Lambda) &= \exp(\frac{1}{2} \omega_{\mu\nu}S^{\mu\nu}) \\
	\psi'_a(x) &= S_{ab}(\Lambda) \psi_b(\Lambda^{-1}x)
\end{align}
\paragraph{adjoint spinor}
\begin{align}
	\bar{\psi} = \psi^\dagger \gamma^0
\end{align}
\paragraph{Fifth gamma matrix}
\begin{align}
	\gamma^5 &\defeq i \gamma^0 \gamma^1 \gamma^2 \gamma^3 \\
	\left\{ \gamma^\mu, \gamma^5 \right\} &= 0 \\
	(\gamma^5)^2 &= \id_4
\end{align}
\paragraph{Plane wavesolutions}
\begin{align}
	\psi(x) = \begin{cases}
		u(p) e^{-ipx} & \text{positive frequency} \\
		v(p) e^{ipx} & \text{negative frequency}
	\end{cases}
\end{align}
\begin{align}
	u_s	(p) = \sqrt{E_p+m} \begin{pmatrix} \chi_s \\ \frac{\pmb{u}\cdot\pmb{p}}{E_p+m}\chi_s\end{pmatrix} e^{-ipx}
	v_s	(p) = \sqrt{E_p+m} \begin{pmatrix} \frac{\pmb{u}\cdot\pmb{p}}{E_p+m}\tilde{\chi}_s \\ \tilde{\chi}_s \end{pmatrix} e^{ipx}
\end{align}
with
\begin{align*}
	\chi_{\frac{1}{2}} = \begin{pmatrix} 1 \\ 0 \end{pmatrix} \quad x_{-\frac{1}{2}} = \begin{pmatrix} 0 \\ 1\end{pmatrix} \\
	\quad s=\pm\frac{1}{2} \quad \tilde{\chi}_s = \chi_{-s}
\end{align*}
\paragraph{Orthogonality of spinor}
\begin{align}
	\bar{u}_s(p) u_{s'}(p) &= -\bar{v}_s(p) v_{s'}(p) = 2m \delta_{ss'} \\
	\bar{u}_s(p) v_{s'}(p) &= 0
\end{align}
\paragraph{Spin sums}
\begin{align}
	\sum_{s} u_s(p)\bar{u}_s(p) &= \slashed{p} + m  \\
	\sum_{s} v_s(p)\bar{v}_s(p) &= \slashed{p} - m 
\end{align}

\section{Dirac Lagrangian and quantization}
\begin{align}
	\lag = \bar{\psi}(x) (i\slashed{\partial} - m) \psi(x)
\end{align}
\paragraph{Quantization}
\begin{align}
	\left\{ \psi_a(\pmb{x}), \psi_b^\dagger(\pmb{x}') \right\} = \delta_{ab}\delta^{(3)}(\pmb{x}-\pmb{x}') \\
	\left\{ \psi_a(\pmb{x}), \psi_b(\pmb{x}') \right\} = \left\{ \psi^\dagger_a(\pmb{x}), \psi^\dagger_b(\pmb{x}') \right\} = 0
\end{align}
\paragraph{Field operators}
\begin{align}
\psi(\pmb{x}) = \int \frac{\dd^3 p}{(2\pi)^3 \sqrt{2E_p}} \sum_s (a_{\pmb{p}}^s u_s(\pmb{p})e^{i\pmb{p}\cdot\pmb{x}} + b_{\pmb{p}}^{s\,\dagger} v_s(\pmb{p})e^{-i\pmb{p}\cdot\pmb{x}})
\end{align}
thus the anticommutations of ladder operators:
\begin{align*}
	\left\{ a_{\pmb{p}}^s, a_{\pmb{p'}}^{s'\,\dagger} \right\} &= \left\{ b_{\pmb{p}}^s, b_{\pmb{p'}}^{s'\,\dagger} \right\}= (2\pi)^3\delta_{ss'}\delta^{(3)}(\pmb{p}-\pmb{p}') \\
	\left\{ a, a \right\} &= \left\{ a^\dagger, a^\dagger \right\} = \dots = 0
\end{align*}
\paragraph{Hamiltonian in terms of Fourier modes (with normal ordering)}
\begin{align}
	H = \int \frac{\dd^3 p}{(2\pi)^3} \sum_s E_p (a_{\pmb{p}}^{s\,\dagger}a_{\pmb{p}}^{s}-b_{\pmb{p}}^{s\,\dagger}b_{\pmb{p}}^{s})
\end{align}
\section{Particles and antiparticles}
\begin{align}
	Q &= e \int \dd^3 x \psi^\dagger(x)\psi(x) \\
	:Q: &= e \int \frac{\dd^3 p}{(2\pi)^3} \sum_s (a_{\pmb{p}}^{s\,\dagger}a_{\pmb{p}}^s - b_{\pmb{p}}^{s\,\dagger}b_{\pmb{p}}^s)
\end{align}
\section{Dirac propagator and anticommutators}
\begin{align}
	\begin{split}
	S_{ab}(x-y) &= \left\{ \psi_a(x), \bar{\psi}_b(y) \right\} \\
				&= (i\slashed{\partial}+m) \left[ D(x-y)-D(y-x) \right]
	\end{split}
\end{align}
\paragraph{Time ordering of Dirac fields}
\begin{align}
	T(\phi_a(x)\bar{\psi}_b(y)) = \Theta(x^0 - y^0)\psi_a(x)\bar{\psi}_b(y) - \Theta(y^0 - x^0) \bar{\psi}_b(y)\psi_a(x)
\end{align}
\paragraph{Feynman propogator for the Dirac field}
\begin{align}
	S_F(x-y) = \braket{0| T \psi(x)\bar{\psi}(y) |0} = \int \frac{\dd^4 p}{(2\pi)^4} \frac{i(\slashed{p}+m)}{p^2-m^2+i\epsilon} e^{-ip\cdot(x-y)}
\end{align}

\section{Discrete symmetries of the Dirac Field}
\begin{center}
\begin{tabular}{c c c}
	\toprule
& orientation perserving & orientation not perserving \\
\midrule
	(ortho)chronous & $\mathcal{L}_+^{\uparrow}$ & $\mathcal{L}_-^\uparrow=\mathcal{P}\mathcal{L}_+^{\uparrow}$ \\
	non-orthochronous & $\mathcal{L}_-^\downarrow = \mathcal{T}\mathcal{L}_+^{\uparrow}$ & $\mathcal{L}_+^{\downarrow}=\mathcal{P}\mathcal{T}\mathcal{L}_+^{\uparrow}$ \\
	\bottomrule
\end{tabular}
\end{center}


\chapter{Interacting QFT}
\section{Introduction and examples}
Theories discussed so far are Klein-Gordon theory (spin $0$) $$\lag_{KG} = \frac{1}{2}\partial_\mu\phi\partial^\mu\phi - \frac{m^2}{2}\phi^2$$ and Dirac theory (spin $\frac{1}{2}$) $$\lag_D = \bar{\psi}(i\slashed{\partial} - m) \psi $$

There is also $\lag_{EM} = -\frac{1}{4}F_{\mu\nu}F^{\mu\nu}$ with $F_{\mu\nu} = \partial_\mu A_\nu - \partial_\nu A_\mu$ for a massless vector filed. Its quantiasation gives photon

One thing they have in common is quadratic in the fields. As result:
\begin{itemize}
	\item linear field equations
	\item exact quantisation
	\item multi-particle states without scattering or interaction
	\item linear fourier decompositions , no mementum changes
\end{itemize}

To have an interacting theory with scattering, need higher powers in the field in the Lagrangians. A few examples are following
\paragraph{scalar $\phi^4$ theory}
\begin{align}
	\lag = \lag_{KG} + \frac{\lambda}{4!} \phi^4
\end{align}
need positive sign $\lambda > 0$ for a stable theory, otherwise classical energy can be arbitarily negative.

Equation of motions
\begin{align}
	(\partial^2+ m ^2) \phi = -\frac{\lambda}{3!} \phi^3
\end{align}
is nonlinear, cannot be solved by Fourier decomposition.

\paragraph{Yukawa-theory}
\begin{align}
	\lag = \lag_{KG} + \lag_{D} - g \bar{\psi}\psi \phi
\end{align}
It is originally developed as a theory for nuclear forces with $\psi$ nucleon, $\phi$ pion. In the Standard Model it is similar to interactions in Higgs mechanism.
\paragraph{Quantum Electrodynamics (QED)}
\begin{align}
	\lag = \lag_{EM} + \lag_{D} - eA_\mu \bar{\psi} \gamma^\mu \psi
\end{align}
descreibes electrons, their antiparticles positrons and photons.

\paragraph{Yang-Mills theory}
generalises $\lag_{EM}$ with terms like $A^4$ or $A^2 \partial A$
\paragraph{Scalar QED}
descreibes pions and photons
\begin{align}
	\begin{split}
		\lag &= \lag_{EM} + D_\mu \phi D^\mu\phi^* - m^2 |\phi|^2 \\
			 &= \lag_{EM} + \partial_\mu\phi\partial^\mu\phi^* - m^2 \phi\phi^* + ie A_\mu(\phi\partial^\mu\phi^* - \phi^* \partial^\mu\phi) + e^2 A_\mu A^\mu \phi\phi^*
	\end{split}
\end{align}
\paragraph{Remarks}
\begin{enumerate}
	\item Interaction terms in $H_\text{int} = \int \dd^3 \ham_\text{int} = - \int\dd^3x\lag_\text{int}$ always involves products of fields at the same point $\pmb{x}$. It ensures causality, no "instant at a distance".
	\item There are no derivative interactions. These may complicate quantisation as $$\pi(\pmb{x}) = \frac{\partial\lag}{\partial(\partial_0 \phi(\pmb{x}))}$$
	\item Why the examples above? There must be zillions of theories (Lagrangians)? \\
			We have the criterion of \textbf{renormalizability}. Note the mass dimensions of fields;
			\begin{align*}
				[S] = 1 \, \text{so} \, [\lag] = [M]^4 \, \Rightarrow [\phi] = [M] ,\, [\psi] = [M]^{\frac{3}{2}} ,\, [A_\mu] = [M]
			\end{align*}
			So in all the interaction terms indicated above, the coupling constant $\lambda$, e, g are all \textbf{dimensionless}!\\
			Can add $-\frac{\mu}{3!}\phi^3$ to the $\phi^4$ theory. This leads to $[\mu] = [M]$ and all these generate renormalisable interactions. \\
			All higher interaction terms require coupling constants of \textbf{negative} mass dimension. e.g. $G\bar{\psi}\psi\bar{\psi}\psi$ and then $[G] = [M]^{-2}$. These are nonrenormalisable and create trouble when performing higher-order calculation in perturbation theory.
			(with energy cutoff; corrections $~G\Lambda^2$, $\Lambda \rightarrow \infty$)
		\item we haven't quantised the photon yet. The reason is that its is a vector field, i.e. 4 degrees of freedom, but photon has just $2$ physical polarisaion states. It is linked to gauge symmetry and complicates quantisation somewhat.
\end{enumerate}
\section{The interaction picture}
Consider the $\phi^4$ theory, 
\begin{align}
\lag_{int} = -\frac{\lambda}{4!} \phi(x)^4
\end{align}
Hamiltonian $H = H_0 + H_{int}$ with 
\begin{align}
	H_0 &= \int\dd^3 x \left\{ \frac{1}{2}\pi^2(x) + \frac{1}{2}(\pmb{\nabla}\phi)^2 + \frac{1}{2}m^2\phi^2 \right\}\\
	H_{int} &= -\int\dd^3x \lag_{int} = \frac{\lambda}{4!}\int\dd^3x \phi^4 
\end{align}

Interaction picture means that operators evolve in time using $H_0$ (only), in particular 
\begin{align}
	\phi_I(t,\pmb{x}) = e^{iH_0t}\phi(\pmb{x})e^{-iH_0t}
\end{align}

Time-dependence of the free field, obeys classical equation of motion $\left(\partial^2+m^2 \right)\phi_I(t,\pmb{x}) = 0$. Solution in terms if fourier modes as before:
\begin{align}
	\phi_I = \int \frac{\dd^3 p}{(2\pi)^3\sqrt{2E_p}} (a^I_{\pmb{p}}e^{-ipx} + a^{I\,\dagger}_{\pmb{p}}e^{+ipx})
\end{align}
as in the free theory with standard commutation relations $[a^I_{\pmb{p}}, a^{I\,\dagger}_{\pmb{p}}] = (2\pi)^3\delta^{(3)}(\pmb{p}-\pmb{p}')$. The state satisfing $a^I_p \ket{0} = 0$ is the vacuum of the free, noninteratcting theory.

Relation between interaction and Schrödinger picure states:
\begin{align}
	\ket{\phi_I(t)} = e^{iH_0t}{\ket{\psi_S(t)}}
\end{align}
Schrödinger equation becomes: 
\begin{align}
	i\frac{\partial}{\partial t}\ket{\psi_S} &= (H_0 + H_\text{int})\ket{\psi_S} \notag\\
	\text{LHS} &= i \frac{\partial}{\partial t} ( e^{-iH_0t}\ket{\phi_I}) = H_0 e^{-iH_0t}\ket{\phi_I} + e^{-iH_0t}i\frac{\partial}{\partial t}\ket{\phi_I} \notag\\
	\text{RHS}														&= \left(H_0 + H_\text{int}\right)e^{-iH_0t}\ket{\phi_I} \notag\\
	\Rightarrow i\frac{\partial}{\partial t}\ket{\phi_I} &= e^{iH_0t}H_{int} e^{-iH_0t} = H_I(t)\ket{\phi_I} \label{math:int-schr}
\end{align}
with $H_I$ interaction Hamiltonian in the interaction picture. Clearly
\begin{align}
	H_I = \frac{\lambda}{4!}\int\dd^3x\phi_I^4(x) \notag
\end{align}

What is the solution of \ref{math:int-schr} for the time evolution of $\ket{\phi_I(t)}$? Define time-evolution operator in the interaction picture.
\begin{align}
	\ket{\phi_I(t)} &= U(t,t_0)\ket{\phi_I(t_0)}\label{math:int-schr2}\\
	\text{where} \; U(t,t_0) &= e^{iH_0(t-t_0)}e^{-iH(t-t_0)} 
\end{align}

With \ref{math:int-schr} and \ref{math:int-schr2}:
\begin{align}
	i\frac{\partial}{\partial t} U(t,t_0) = H_I(t)U(t,t_0)
\end{align}

To solve with boundary conditions: $U(t_0, t_0) = \id$. The formal solution:
\begin{align}
	U(t,t_0) = 1 - i \int_{t_0}^t \dd t' H_I(t')U(t',t_0) \notag
\end{align}

Substitute back in and we get:
\begin{align}
	U(t,t_0) = 1- i \int_{t_0}^t \dd t' H_I(t') + (-i)^2 \int_{t_0}^t \dd t' \int^{t'}_{t_0} \dd t'' H_I(t') H_I(t'') + \dots
\end{align}

Ranges of integration: $H_I$ in the product is automatically time-ordered.

\begin{figure}[ht]
	\centering
	\includegraphics[width=0.65\linewidth]{4-2-triangle.jpg}
	\caption{Time ordering}
	\label{fig:4-2-triangle}
\end{figure}

Upper triangle has the wrong time order. We are going to "repair" it by hand.
\begin{align}
	U(t,t_0) &= 1-i\int_{t_0}^t \dd t' H_I(t') + \frac{(-i)^2}{2} \int_{t_0}^t \dd t' \int^{t'}_{t_0} \dd t'' T(H_I(t') H_I(t'')) + \dots \notag\\
			 &= \sum_{n=0}^{\infty} \frac{(-i)^n}{n!}\int_{t_0}^t \dd t_1 \dots \int_{t_0}^t \dd t_n T(H_I(t_1) \dots H_I(t_n))  \notag\\
			 &= T \exp{-i\int_{t_0}^t \dd t' H_I(t') }
\end{align}

It is interesing for scattering to transition into asymptotic state for $t \rightarrow \infty$
\begin{align}
	S = \lim_{t \rightarrow \infty} U(t,-t) &= T \exp{-i \int_{-\infty}^\infty \dd t H_I(t)} \\
	 &\stackrel{\phi^4}{=} T \exp{-i\int \dd^4 x \frac{\lambda}{4!} \phi_I^4(x)} \notag
\end{align}
Both $U$ and $S$ are formally unitary

Composition law for time evolution operator: 
\begin{align}
	U(t_2, t_0) = U(t_2, t_1) U(t,t_0) = U(t_2, t_1) U(t_0, t_1)^\dagger
\end{align}

\subsection{Scattering amplitudes and the S-matrix}
Take $\ket{i}$ the initial (multi-particle) state and $\ket{f}$ the final (multi-particle) state. Time evolution of $\ket{i}$ then is
$$\lim(t\rightarrow\infty)U(t,-\infty) \ket{i} = S \ket{i}$$

Probability that $\ket{i}$ evolves into $\ket{f}$ is proporional to the squared "\textbf{S-matrix element}"
\begin{align}
	\left|	\braket{f,\, t\to \infty | i, t\rightarrow - \infty}\right|^2 = \left| \braket{f|S|i} \right|^2 = \left| S_{fi} \right|^2
\end{align}

The nontrivial part of the S-matrix is the T-matrix:
\begin{align}
	S_{fi} \defeq \delta_{fi} + iT_{fi}
\end{align}

Use momentum conservation (from translation invariance) to define matrix element
\begin{align}
	S_{fi} = \delta_{fi} + i(2\pi)^4 \delta^{(4)}(p_f - p_i) M_{fi}
\end{align}
$M_{fi}$ measures "genuine scattering" from $\ket{i}$ to $\ket{f}$.

How are we going to calculate correlation functions in the interacting theory:
\begin{align}
	\braket{\Omega |  \ T \phi(x)\phi(y) | \Omega}	
\end{align}
or more generally $\braket{\Omega | T \phi(x_1)\phi(x_2)\dots | \Omega}$, where $\ket{\Omega}$ is the vaccum/ground state of the interacting theory and $\phi(x)$ the Heisenberg operators.

Ignore $\ket{\Omega} \neq \ket{0}$ for the moment other than saying: we want to study the time evolution from the vacuum at $t\rightarrow - \infty$ to $t \rightarrow + \infty$. So rewriting in terms $\phi_I(x)$, assuming $x^0 > y^0$ for now:
\begin{align}
	\braket{0 | U(\infty, x^0) \phi_I(x^0)  U(x^0, y^0) \phi_I(y^0)U(y^0, -\infty) | 0}  = \braket{0|T(\phi_I(x)\phi_I(y)S)|0}
\end{align}
still holds if $x^0 < y^0$ because of $T$.

Now $\ket{\Omega} \neq \ket{0}$: this can be taken care of by dividing out the time evolution of the (free) vacuum $\braket{0|S|0}$, so
\begin{align}
	\braket{\Omega | T(\phi(x)\phi(y))|\Omega} \notag\\
	&= \frac{\braket{0 | T(\phi_I(x)\phi_I(y)S)|0}}{\braket{0|S|0}} \\
	&\stackrel{\phi^4}{=} \frac{
		\braket{0 | T {\phi_I(x)\phi_I(y)\exp{-i\int\dd^4x' \frac{\lambda}{4!}\phi^4(x')}}| 0}
		}{
		\braket{0 | T {\exp{-i\int\dd^4x' \frac{\lambda}{4!}\phi^4(x')}}| 0}
}\notag
\end{align}
Proof can be found in Peskin. It will also be illustrated parctically later ("vacuum bubbles").

Perturbation theory is viable when $\lambda$ (or some other coupling) is "small" and then expands $U(t,t_0)$ or $S$ in powers of $\lambda$.

\section{Wick's theorem}
From now on drop the subscript for interaction pictire fields $\phi_I(x) \rightarrow \phi(x)$.

Want to calculate stuff like $\braket{0 | T\phi(x_1)\dots\phi(x_n)S | 0}$ in pert. theory; so e.g. at order $\lambda^n$. So 
\begin{align}
\frac{1}{n!} \left( -i\frac{\lambda}{4!} \right)^n \int \dd^4 y_1 \dots \dd^4 y_n \braket{0 | T\phi(x_1)\dots\phi(x_n)\phi^4(y_1)\dots\phi^4(y_n) | 0}
\end{align}
is tough!

We know $\braket{0 | T\phi(x_1)\phi(x_2) | 0 } $ is the Feynman propagator!

Recall \textbf{normal ordering} with $\phi(x) = \phi^+(x) + \phi^-(x)$
\begin{align}
	:\phi^+ \phi^-: = :\phi^- \phi^+: = \phi^- \phi^+
\end{align} 

Wick's therem expresses time-ordered products in terms of normal-ordered ones. Then it is easy to take vacuum expectation values, as $\braket{0 | :\phi(x_1)\dots\phi(x_n):|0} = 0$

Take two fields and $x^0 > y^0$:
\begin{align*}
	T \phi(x)\phi(y) &= \phi(x)\phi(y) = \left(\phi^+(x)+\phi^-(x)\right)\left(\phi^+(y)+\phi^-(y)\right) \\
	&= \phi^+(x)\phi^+(y) + \phi^-(x)\phi^-(y) + \phi^-(x) \phi^+(y) + \phi^-(x) \phi^+(y) + [\phi^+(x), \phi^-(y)] \\
	&= :\phi(x)\phi(y): + [\phi^+(x),\phi^-(y)]
\end{align*} 

Particularly for $y^0 > x^0$: 
\begin{align}
	T\phi(x)\phi(y) = :\phi(x)\phi(y): + [\phi^+(y),\phi^-(x)] \notag
\end{align}

Thus altogether: 
\begin{align}
	T\phi(x)\phi(y) = :\phi(x)\phi(y): + D_f(x-y)
\end{align}
as $\Theta(x^0 - y^0) [\phi^+(x), \phi^-(y)] + \Theta(y^0 - x^0) [\phi^+(y), \phi^-(x)] = D_F(x-y)$.

Worth noting that $D_F(x-y)$ is still a c-number, not operator (yet). Thus it can be pulled out of any matrix element or expectation value.

We now define "contraction":
\begin{align}
	\contraction[1ex]{}{\phi}{(x_1)}{\phi}	\phi(x_1) \phi(x_2) = D_F(x_1-x_2)
\end{align}

Thus we can remove the fields from the product leaving only the propagators:
\begin{align}
	T\phi(x)\phi(y) = :\phi(x)\phi(y): +  \contraction[1ex]{}{\phi}{(x)}{\phi}	\phi(x) \phi(y)
\end{align}

General form of \textbf{Wick's theorem} for arbitary number of fields
\begin{align}
	T\phi(x_1)\dots \phi(x_n) = :\phi(x_1)\dots \phi(x_n):	 + :\left( \text{sum over all possible contractions} \right):
\end{align}

Example with four fields:
\begin{align*}
	T(\phi_1 \phi_2 \phi_3 \phi_4) &= :\phi_1 \phi_2 \phi_3 \phi_4: \\
								   & + \contraction{}{\phi_1}{}{\phi_2} \phi_1 \phi_2 :\phi_3 \phi_4: + \contraction{}{\phi_1}{}{\phi_3} \phi_1 \phi_3 :\phi_2 \phi_4: + \contraction{}{\phi_1}{}{\phi_4} \phi_1 \phi_4 :\phi_2 \phi_3: +  \contraction{}{\phi_2}{}{\phi_3} \phi_2 \phi_3 :\phi_1 \phi_4: + \contraction{}{\phi_2}{}{\phi_4} \phi_2 \phi_4 :\phi_1 \phi_3: + \contraction{}{\phi_3}{}{\phi_4} \phi_3 \phi_4 :\phi_1 \phi_2: \\
	&+ \contraction{}{\phi_1}{}{\phi_2} \phi_1 \phi_2 \contraction{}{\phi_3}{}{\phi_4}\phi_3 \phi_4 + \contraction{}{\phi_1}{}{\phi_3} \phi_1 \phi_3 \contraction{}{\phi_2}{}{\phi_4}\phi_2 \phi_4 + \contraction{}{\phi_1}{}{\phi_4} \phi_1 \phi_4 \contraction{}{\phi_2}{}{\phi_3}\phi_2 \phi_3
\end{align*}

Thus 
\begin{align*}
	\braket{ 0 | T \left(\phi_1 \phi_2 \phi_3 \phi_4 \right) | 0} = D_F (x_1 - x_2) D_F(x_3 - x_4) + D_F (x_1 - x_3) D_F(x_2 - x_4) + D_F (x_1 - x_4) D_F(x_2 - x_3) 
\end{align*}
which can be visually represented as
\begin{figure}[ht]
	\centering
	\includegraphics[width=0.7\linewidth]{4-3-feynman.jpg}
	\caption{Feynman diagrams}
	\label{fig:4-3-feynman}
\end{figure}

\chapter{Quantum Electrodynamics (QED)}
\section{Classical Electrodynamics and Maxwell's equations}
We have the gauge potential $A^\mu = (A^0, \pmb{A}) = (\phi, \pmb{A})$ \& $A_\mu = (A^0, -\pmb{A}) = (\phi, -\pmb{A})$ and the field strength tensor $F_{\mu\nu} = \partial_\mu A_\nu - \partial_\nu A_\mu$.

Then
\begin{itemize}
	\item electric field $E_i = F_{0i} = \partial_0 A_i - \partial_i A_0 \rightarrow \pmb{E} = -\dot{\pmb{A}} - \pmb{\nabla} \phi$
	\item magnetic field $B^i = -\frac{1}{2} \epsilon^{ijk}F_{jk} \rightarrow \pmb{B} = \pmb{\nabla} \times \pmb{A}$
\end{itemize}

Lagrangian density $\lag_{EM} = -\frac{1}{4}F_{\mu\nu}F^{\mu\nu} = -\frac{1}{2} (\pmb{E}\cdot\pmb{E} - \pmb{B}\cdot \pmb{B})$. The field equation $\partial_\mu \left( \frac{\partial \lag}{\partial(\partial_\mu A_\nu)} \right) - \frac{\partial \lag}{\partial A_\nu} = 0$ leads to
\begin{align}
	\partial_\mu F^{\mu\nu} = 0
\end{align}
it is half of Maxwell's equations (in vacuum).

The other half are Bianchi identities following from the definition of $F_{\mu\nu}$:
\begin{align*}
	\partial_\lambda F_{\mu\nu} + \partial_\mu F_{\nu\lambda} + \partial_\nu F_{\lambda\mu} = 0 \Leftrightarrow \epsilon^{\sigma \lambda \mu \nu}\partial_\lambda F_{\mu\nu} = 0\\
	\text{or } \partial_\lambda \tilde{F}^{\sigma\lambda} = 0,\; \tilde{F}^{\sigma\lambda} = \frac{1}{2} \epsilon^{\sigma\lambda\mu\nu}F_{\mu\nu}
\end{align*}

In terms of $\pmb{E}$ and $\pmb{B}$:
\begin{align*}
	\pmb{\nabla}\cdot \pmb{E} = 0&,\; \dot{\pmb{E}} = \pmb{\nabla}\times \pmb{B} \quad \text{dynamical equations} \\
	\pmb{\nabla}\cdot \pmb{B} = 0&,\; \dot{\pmb{B}} = -\pmb{\nabla}\times \pmb{E} \quad \text{Bianchi identities}
\end{align*}

Remarks
\begin{itemize}
	\item Lagrangian density does not depend on $\dot{A}_0$, since $A_0$ is not really dynamical.
	\begin{align*}
		\pmb{\nabla}\cdot \pmb{E} = 0 \rightarrow \pmb{\nabla}^2 A_0 + \pmb{\nabla}\cdot \dot{\pmb{A}} = 0 
	\end{align*}
	Solve this \underline{Poisson} equation for $A_0(\pmb{x},t) = \frac{1}{4\pi}\int \dd^3 y \frac{\pmb{\nabla}\cdot \dot{\pmb{A}}(\pmb{y},t)}{|\pmb{y}-\pmb{x}|}$. Thus $A_0$ is given in terms of the other components of $A$.
	\item \underline{gauge invariance}: field strength tensor invariant under the transformation $A_\mu \longmapsto A_\mu - \partial_\mu \text{X}$ due to commuting derivatives. This leads to gauge invariance of Maxwell equations.\\
	Choose X to satisfy $\partial_\mu \partial^\mu \text{X} = \partial^2\text{X}=\partial_\mu A^\mu$ allows us to demand the condition  (Lorenz condition)
		\begin{align}
			\partial_\mu A^\mu = 0
		\end{align}
	such that $A_\mu$ belongs to the "Lorenz gauge" and reduces the degrees of freedom from 4 to 3.
	\begin{itemize}
		\item Further freedom is eliminated by adding any X with $\partial^2 \text{X} = 0$, e.g.~$\frac{\partial}{\partial t} \text{X} = A_0$. Then we get the \underline{Coulomb} or \underline{radiation gauge}
		\begin{align}
			A_0 = 0,\; \pmb{\nabla}\cdot \pmb{A} = 0
		\end{align}
	\end{itemize}
\end{itemize}

Note: vice versa imposing $\pmb{\nabla} \cdot \pmb{A} = 0$ first, yields $A_0 = 0$ (using Lorenz condition?).

In Coulomb gauge:
\begin{align*}
	&\pmb{E} = -\dot{\pmb{A}}.\; \pmb{B} = \pmb{\nabla} \times \pmb{A},\; \pmb{\nabla}\times \pmb{A} = 0 \\
	&-\ddot{\pmb{A}} = \dot{\pmb{E}} \stackrel{\text{Maxwell}}{=} \pmb{\nabla} \times \pmb{B} =  \pmb{\nabla} \times (\pmb{\nabla} \times \pmb{A}) = \pmb{\nabla}(\underbrace{\pmb{\nabla}\cdot\pmb{A}}_{=0}) - \pmb{\nabla}^2 \pmb{A} \\
	&\Rightarrow \partial^2 \pmb{A} = 0
\end{align*}
This wave equation is massless KG equation for each spatial component.

Then the solutions are obvious: $\pmb{A} = \pmb{\epsilon} e^{-ik\cdot x}$ with $k^2=0$ and $\pmb{\epsilon}\cdot\pmb{k}=0$. The polarization vector $\pmb{\epsilon}$ is transverse to $\pmb{k}$.

Can write the lagrangian in Coulmb gauge 
\begin{align*}
	\lag_\text{EM} = \frac{1}{2}\dot{\pmb{A}}\dot{\pmb{A}} - \frac{1}{2} \pmb{B}\cdot \pmb{B}
\end{align*}
Then the conjugate momentum to $\pmb{A}$ is $\pmb{\Pi} = \frac{\partial \lag}{\partial \dot{\phi}} = \dot{\pmb{A}} = -\pmb{E}$. (only 3 components, there is no conjugate momentum to $A_0$!) $\pmb{\Pi}$ is subject to the constraint $\pmb{\nabla}\cdot\pmb{\Pi} = 0$

Hamiltonian
\begin{align*}
	H_{\text{EM}} = \int \dd^3 x \left( \frac{1}{2} \pmb{\Pi}\cdot\pmb{\Pi} + \frac{1}{2} \pmb{B}\cdot\pmb{B} \right)
\end{align*}

\section{Quantizing the Maxwell field}
We would like to impose canonical commutation relations, à la
\begin{align*}
	&[ A_i(\pmb{x}), A_j (\pmb{y}) ] =  [ \Pi_i(\pmb{x}), \Pi_j (\pmb{y}) ] = 0 \\
	&[A_i(\pmb{x}), \Pi_j(\pmb{y})] = i \delta_{ij} \delta^{(3)} (\pmb{x} - \pmb{y})
\end{align*}

However this cannot be true. Take either derivative of the last equation and it needs to vanish deu to $\pmb{\nabla}\cdot\pmb{A} = \pmb{\nabla}\cdot \pmb{\Pi} = 0$. But 
\begin{align*}
	[\partial^i A_i (\vecx), \Pi_k(\vecy)] =  i \delta_{ij} \partial^i \delta^{(3)}(\vecx-\vecy)
\end{align*}
here the derivative is takev with respect to $\vecx$, i.e.~$\partial^i = \frac{\partial}{\partial x_i}$.


Replace $\delta_{ij}$ by $\Delta_{ij}$
\begin{align*}
	& [\partial^i A_i (\vecx), {\Pi}_{j}(\vecy)] = i \Delta_{ij} \partial^i \frac{1}{(2\pi)^3} \int \dd^3 k e^{i\pmb{k}\cdot(\vecx - \vecy)} \\
	& = -\frac{1}{(2\pi)^3} \int \dd^3 k (k^i\Delta_{ij}) e^{i\pmb{k}\cdot(\pmb{x}-\pmb{y})} \stackrel{!}{=} 0
\end{align*}
it works for $\Delta_{ij} = \delta_{ij}-\frac{k_ik_j}{\pmb{k}^2}$ in momentum space or $\Delta_{ij} = \delta_{ij} - \vecnab^{-2}\partial_i \partial_j$ in position space.

\begin{align}
	[A_i(\pmb{x}), \Pi_j(\pmb{y})] = i \left( \delta_{ij} - \vecnab^{-2}\partial_i \partial_j \right) \delta^{(3)} (\pmb{x} - \pmb{y})
\end{align}

As before we have the mode expansion
\begin{align*}
	\pmb{A}(\vecx) &= \int \frac{\dd^3 k}{(2\pi)^3\sqrt{2|\veck|}} \left( \pmb{a}_{\veck}e^{i\veck\cdot\vecx} + \pmb{a}^\dagger_{\veck} e^{-i\veck\cdot\vecx} \right) \\
	\pmb{\Pi}(\vecx) &= \int \frac{\dd^3 k}{(2\pi)^3} (-i)\sqrt{\frac{|\veck|}{2}}\left( \pmb{a}_{\veck}e^{i\veck\cdot\vecx} - \pmb{a}^\dagger_{\veck} e^{-i\veck\cdot\vecx} \right) \\
\end{align*}
with $\veck\cdot\pmb{a}_{\veck} = \veck\cdot\pmb{a}^\dagger_{\veck} = 0$.

Introduce 2 orthogonal polarization vectors $\pmb{\epsilon}^{(1)}(\veck)$ and $\pmb{\epsilon}^{(2)}(\veck)$ for each $\veck$.
\begin{align*}
	&\pmb{a}_{\veck} = a_{\veck}^{(1)}\pmb{\epsilon}^{(1)} + a_{\veck}^{(2)}\pmb{\epsilon}^{(2)} = \sum_{\lambda=1}^2 a_{\veck}^{(\lambda)}\pmb{\epsilon}^{(\lambda)}(\veck) \\
	& \text{with } \veck\cdot\pmb{\epsilon}^{(1)}(\veck) = \veck\cdot\pmb{\epsilon}^{(2)}(\veck) = 0,\; \pmb{\epsilon}^{(\lambda)}\cdot \pmb{\epsilon}^{(\lambda;)}=\delta_{\lambda \lambda'}
\end{align*}

Creation and annihilation operator have the standard commutation relations
\begin{align}
	[a_{\veck}^{(\lambda)}, a_{\veck'}^{(\lambda')\dagger}] = (2\pi)^3 \delta_{\lambda\lambda'}\delta^{(3)}(\veck-\veck')
\end{align} 
and all other commutators vanish. Geometrically, still possible to write
\begin{align*}
	[\pmb{a}_{\veck}, \pmb{a}_{\pmb{l}}] &= 	[\pmb{a}_{\veck}^\dagger, \pmb{a}_{\pmb{l}}^\dagger] = 0 \\
	[a^i_{\veck}, a_{\pmb{l}}^{j\dagger}] &= (2\pi)^3 \left( \delta^{ij} - \frac{k^i k^j}{\veck^2} \right) \delta^{(3)}(\veck-\pmb{l})
\end{align*}

$a_{\veck}^{(\lambda)}$ and $a_{\veck}^{(\lambda)\dagger}$ create and destroy photons of momentum $\veck$, energy $|\veck|$ and (electric) polarization along $\pmb{\epsilon}^{(\lambda)}(\veck)$. 

Next steps are analogout to KG theory. 
\paragraph{Hamiltonian}
\begin{align*}
	H &= \frac{1}{2} \int \dd^3 x \left(\pmb{E}^2 + \pmb{B}^2 \right)= \frac{1}{2} \int \dd^3 x \left( \dot{\pmb{A}}^2 + (\vecnab \times \pmb{A})\cdot (\vecnab \times \pmb{A}) \right) \\
	\shortintertext{using identity $\pmb{A}\cdot(\pmb{B}\times\pmb{C}) = \pmb{B}\cdot(\pmb{C}\times\pmb{A})$}
	  &= \frac{1}{2} \int \dd^3 x \left( \dot{\pmb{A}}^2 + \pmb{A}\cdot\vecnab\times(\vecnab \times \pmb{A})\right) \\
	  \shortintertext{using the identity $\vecnab\times(\vecnab\times\pmb{A}) = \vecnab(\vecnab\cdot\pmb{A} - \vecnab^2\pmb{A})$}
	  &= \frac{1}{2} \int \dd^3 x \left( \dot{\pmb{A}}^2 -\pmb{A}\cdot\vecnab^2\pmb{A}+\pmb{A}\cdot\vecnab(\vecnab\cdot\pmb{A})  \right) \\
	\shortintertext{using coulomb gauge condition}
	  &= \frac{1}{2} \int \dd^3 x \left( \dot{\pmb{A}}^2 - \pmb{A}\cdot \nabla^2\pmb{A} \right) \\
	  \shortintertext{the first term vanishes and use normal ordering}
	  &= \int \frac{\dd^3 k}{(2\pi)^3} \left|\veck \right| \pmb{a}_{\veck}^\dagger \cdot \pmb{a}_{\veck} = \sum_{\lambda=1}^2 \int \frac{\dd^3 k}{(2\pi)^3} \left|\veck \right| a_{\veck}^{(\lambda \dagger)} a_{\veck}^{\lambda}
\end{align*} 

\paragraph{Heisenberg field}
\begin{align*}
	\pmb{A}(\pmb{x},t) = \int \frac{\dd^3 k}{(2\pi)^3}\frac{1}{\sqrt{2|\pmb{k}|}} \left( \pmb{a}_{\pmb{k}} e^{-ik\cdot x} +  \pmb{a}_{\pmb{k}}^\dagger e^{ik\cdot x}\right)
\end{align*}

\paragraph{Photon propagator}
\begin{align}
	\braket{0 | T A_i (x) A_j (y) | 0} \eqdef D^{\text{tr}}_{ij} (x-y) = \int\frac{\dd^4 k}{(2\pi)^4}	
\frac{i}{k^2 + i\epsilon} \left( \delta_{ij} - \frac{k_i k_j}{|\pmb{k}|^2} \right)e^{-ik\cdot(x-y)}
\end{align}
$\text{tr}$ stands for transverse: photon polarization perpendicular to its momentum. This is \textcolor{red}{NOT} the final version of the photon propagator!

\section{Inclusion of matter - QED}
\begin{align}
	\lag_\text{QED} &= -\frac{1}{4}F_{\mu\nu}F^{\mu\nu} + \bar{\psi} (i\slashed{D} - m)\psi
	\shortintertext{where $D_\mu = \partial_\mu + ieA_\mu$ is the (gauge) covariante derivative}
					&= \lag_\text{EM} + \lag_{D} -e \underbrace{\bar\psi \gamma^\mu \psi A_\mu}_{j^\mu}
\end{align}

Field equations would be
\begin{align*}
	\partial_\mu F^{\mu\nu} = e j^\nu \qquad (i\slashed{D}-m) \psi = 0
\end{align*}
where $ej^\nu$ is the electromagnetic 4-current.

Gauge invariance under the transformation
\begin{align*}
	\begin{cases}
		\psi(x) \longmapsto \psi'(x) = e^{ie\chi(x)}\psi \\
		A_\mu(x) \longmapsto A'_\mu(x) = A_\mu(x) - \partial_\mu \chi(x)
	\end{cases}
\end{align*}

To check the consistence: cavariant derivative transforms like $D_\mu \longmapsto D'_\mu\psi'(x) = e^{ie\chi(x)}D_\mu \psi(x)$. Since the adjoint spinor transforms like $\bar{\psi}(x) \longmapsto \bar{\psi}'(x) = \bar{\psi}(x) e^{-ie\chi(x)}$, the Lagrangian and field equations are gauge invariant.

Again we choose Coulomb gauge $\vecnab \cdot \pmb{A} = 0$, then equation for $A^0$:
\begin{align}
	\partial_i F^{i0} &= ej^0 \notag\\ 
	\Rightarrow -\vecnab^2 A^0 &= ej^0 =  e \bar{\psi}\gamma^0 \psi \notag\\
							   &= e \bar{\psi}\gamma^0\psi = e\psi^\dagger\psi \notag\\
							   &= e \rho(x)\notag\\
	A^0(\pmb{x},t) &= e \int \dd^3 y \frac{\rho(\pmb{y}, t)}{4\pi \left| \pmb{x} - \pmb{y} \right|  }
\end{align}

\chapter{Radiative corrections}
\setcounter{chapter}{6}
\section{Optical theorem}
We have seen in Advanced Quantum Theory that tree diagrams are in general \underline{real}. So there is no imaginary parts. Need to restore perturbatively in higher-order corrections. Then the optical theorem is valid again.

S-matrix is unitary: $S^\dagger S = \id$ with $S = \id + i T$. Thus
\begin{align*}
	-i (T - T^\dagger) = T^\dagger T
\end{align*}

We take matrix element for $k_1 k_2 \rightarrow p_1 p_2$ scattering. On RHS, insert a complete set of states,
\begin{align*}
	\braket{p_1 p_2 | T^\dagger T | k_1 k_2} = \sum_n \prod_{i=1}^n \int \frac{\dd^3 q_i}{(2\pi)^3 2 E_i} \braket{p_1 p_2 | T^\dagger | q_1 \dots q_n} \braket{q_1 \dots q_n | T | k_1 k_2}
\end{align*}

Reduce $T_{fi} = (2\pi)^4 \delta^{(4)}(p_f - p_i) M_{fi}$ and omitting overal $(2\pi)^4 \delta^{(4)}(p_f - p_i)$
\begin{align*}
	-i \left[ \M(k_1 k_2 \rightarrow p_1 p_2) - \M^* (p_1 p_2 \rightarrow k_1 k_2) \right]& \\
	= \underbrace{\sum \prod_{i=1}^n \int \frac{\dd^3 q_i}{(2\pi)^3 2 E_i}}_\text{invariant phase-space volume element} & \M^*(p_1 p_2 \rightarrow q_1 \dots q_n) \M(k_1 k_2 \rightarrow q_1 \dots q_n) (2\pi)^4 \delta^{(4)}(k_1 + k_2 - \sum_i q_i)
\end{align*}

So optical theorem, for forward scattering ($p_1 = k_1, p_2 =  k_2$) reads (see \ref{math:F})
\begin{align*}
	\Im \M(k_1 k_2 \rightarrow k_1 k_2) &= 2F \sigma_\text{tot} (k_1 k_2 \rightarrow \text{anything}) \\
	2\sqrt{s} |f_i^\text{CMS}| &= \lambda^{\frac{1}{2}} (s, m_1^2, m_2^2)
\end{align*}

\paragraph{Optical theorem for Feynman diagrams}
Consider a specific diagram contributing to the imaginary part, e.g.~in $\phi^4$-theory.
\begin{align}
	\feynmandiagram[small, baseline=(x.base), horizontal=x to y]{
		k1[particle=\(k_1\)] -- x --[quarter left] y -- p1[particle=\(p_1\)];
		k2[particle=\(k_2\)] -- x --[quarter right] y -- p2[particle=\(p_2\)];
	};
	p_s = k_1 + k_2 = p_1 + p_2, p_s^2 = s \notag\\
	i\M(s) = \frac{\lambda^2}{2} \int \frac{\dd^4 q}{(2\pi)^4} \frac{1}{ \left[ (p_s /2 - q)^2 - M^2 +i \epsilon \right] \left[ (p_s /2 + q)^2 - M^2 + i\epsilon \right] }\label{math:M}
\end{align}

From optical theorem: $\Im \M(s < 4M^2) = 0$, so $\M(s < 4M^2) \in \R$, (Since it is physical case, the cross section must vanish) when regarding $\M(s)$ as an analytic function of $s$ beyond what physical S-matrix element allow.

\paragraph{Schwarz reflection principle}
If (in some region) analytic function $\M(s)$ is \underline{real} at least for a finite, nonvanishing interval $\in \R$, then
\begin{align}
	\M(s^*) = \M^* (s)
\end{align}

Hence 
$$\M(s+i\epsilon)-\M(s-i\epsilon) \equiv \text{disc} \M(s) = \M(s + i\epsilon) - \M^*(s+i\epsilon) = 2i \Im \M(s+i\epsilon)$$
\begin{center}
\tikzset{zigzag/.style={decorate, decoration=zigzag}}
\begin{tikzpicture}[scale=1, transform shape]
	\draw [thick, <->] (0,3) -- (0,0) -- (3,0);
	\node [above] at (0,3) {$\Im(s)$};
	\node [right] at (3,0) {$\Re(s)$};
	\draw [fill] (1,0) circle [radius=0.05];
	\node [below] at (1,0) {$4M^2$};
	\draw [thick, zigzag, red] (1,0) -- (3,0);
	\draw [thick, ->] (2,1) -- (2,0.1);
	\draw [thick, ->] (2,-1) -- (2,-0.1);
\end{tikzpicture}
\end{center}

Onset of imaginary part for $s \leq 4M^2$ necessarily leads to a "branch cut", a nontrivial discontinuity in the comlex energy plane. The branch cut is equivalent to $\sqrt{4M^2 - s}$. Function has discontinuity, a cut, on real axis.

How can we calculate the discontinuity ($=$ imaginary part) of the above diagram?

Use centre-of-mass system $p_s = (\sqrt{s}, \pmb{0})$. Poles from propagators 
\begin{align*}
	& \frac{s}{4} \mp \sqrt{s}q^0 + q^2 - M^2 + i\epsilon = 0  \\
	\Leftrightarrow &(q^0)^2 \pm \sqrt{s} q^0 + \frac{s}{4} - |\pmb{q}|^2 -M^2 + i\epsilon = 0
\end{align*}

\begin{align*}
	& \text{first propagator} && q^0 = + \frac{\sqrt{s}}{2} \pm (\sqrt{M^2 + |\pmb{q}|^2} - i\epsilon) = +\frac{\sqrt{s}}{2} \pm (E_q - i\epsilon) \\
	& \text{second propagator} && q^0 = -\frac{\sqrt{s}}{2} \pm (E_q - i\epsilon)
\end{align*}

\begin{center}
\tikzset{zigzag/.style={decorate, decoration=zigzag}}
\begin{tikzpicture}[scale=1, transform shape]
	\draw [thick, <->] (0,2) -- (0,0) -- (5,0);
	\draw [thick] (0,-2) -- (0,0) -- (-5,0);
	\node [above] at (0,2) {$\Im(q^0)$};
	\node [right] at (5,0) {$\Re(q^0)$};
	\node [above] at (-2,0.5) {$+\frac{\sqrt{s}}{2} - E_q + i\epsilon$};
	\draw [fill] (-2,0.5) circle [radius=0.05];
	\node [above] at (-5,0.5) {$-\frac{\sqrt{s}}{2} - E_q + i\epsilon$};
	\draw [fill] (-5,0.5) circle [radius=0.05];
	\node [below] at (2,-0.5) {$-\frac{\sqrt{s}}{2} + E_q + i\epsilon$};
	\draw [fill] (2,-0.5) circle [radius=0.05];
	\node [below] at (5,-0.5) {$+\frac{\sqrt{s}}{2} + E_q + i\epsilon$};
	\draw [fill] (5,-0.5) circle [radius=0.05];
\end{tikzpicture}
\end{center}

If we close the contour of the $q_0$ integration in the \underline{lower} half plane, we only pick up the 2 residues at $\mp \frac{\sqrt{s}}{2}+E_q -i\epsilon$. As $E_q$ is positive, only $-\frac{\sqrt{s}}{2} + E_q -i \epsilon$ from second propagator contirbutes to discontinuity. So pinching up the residue equivalent to replacement under $q^0$ integration
\begin{align*}
	\frac{1}{(p_s /2 + q)^2 - M^2 +i\epsilon} \longmapsto  \underbrace{-2\pi i}_{\text{orientation of contour}} \delta((p_s / 2 + q)^2 - M^2)
\end{align*}

Determine the residue of the rest at the pole at $-\frac{\sqrt{s}}{2} + E_q  - i\epsilon$
\begin{align}
	M(s) \longmapsto &-\frac{\lambda^2}{2} \int \frac{\dd^3 q}{(2\pi)^3} \frac{1}{2E_q \sqrt{s}(\sqrt{s}-2E_q)} \notag \\
	\shortintertext{With no angular dependence and using substitution (note the limits of integral also change)$\dd^3 q \rightarrow 4 \pi |\pmb{q}|^2 \dd |\pmb{q}| = 4 \pi |\pmb{q}|E_q \dd E_q$}
	& = -\frac{\lambda^2}{8\pi^2} \int^\infty_M \frac{\dd E_q \sqrt{E_q^2 - M^2}}{\sqrt{s}(\sqrt{s} - 2E_q)}  \label{math:res}
\end{align}

It has pole at $E_q = \frac{\sqrt{s}}{2}$. The second pole in \ref{math:M} at $\frac{\sqrt{s}}{2} + E_q - i\epsilon$ would produce a pole in \ref{math:res} for $E_q = -\frac{\sqrt{s}}{2}$, outside the integration range $M \leq E_q < \infty$.

\begin{itemize}
	\item for $\sqrt{s} < 2M$, \ref{math:res} is manifestly real.
	\item for $\sqrt{s} > 2M$, the pole at $E_q = \frac{\sqrt{s}}{2}$ in \ref{math:res} contributes \underline{differently} depending on $\sqrt{s}\pm i\epsilon$; difference yields discontinuity.
\end{itemize}
Use
\begin{align*}
	\frac{1}{\sqrt{s}-2E_q\pm i\epsilon} = \underbrace{\frac{P}{\sqrt{s} - 2 E_q}}_{\text{real}} \underbrace{\mp i\pi \delta(\sqrt{s} - 2 E_q)}_{\text{yields discontinuity}}
\end{align*}

So for calculation of the discontinuity, have replacement 
\begin{align*}
	\frac{1}{(p_s/2 - q)^2 - M^2 + i\epsilon} \longmapsto -2\pi i \delta((p_s/2 -q)^2 - M^2)
\end{align*}
for other propagator too!

\paragraph{Cuthosky rules (1960)} replace cut propagator according to 
\begin{align}
	\frac{1}{p^2 - M^2 + i\epsilon} \longmapsto -2\pi  i \delta(p^2 - M^2)
\end{align}
to calculate discontinuity across the cut!

Calculateion completed:
\begin{align*}
	\text{disc} \left(	
		\feynmandiagram[small, baseline=(x.base), horizontal=x to y]{
			k1 -- x --[quarter left] y -- p1;
			k2 -- x --[quarter right] y -- p2;
		};
	\right )
	 &= i\frac{\lambda^2}{2} \int \frac{\dd^4 q}{(2\pi)^4} 2\pi \delta(q^2 - M^2) 2\pi \delta((p_s - q)^2 - M^2) \\
	 \shortintertext{using $\dd^4 q = \dd q^0 \dd q |q|^2 \dd \Omega_q$ and $(p_s - q)^2 - M^2 = s - 2\sqrt{s}q^0$}
	 &= \frac{\lambda^2}{2} \frac{i}{4\pi^2} \int \frac{|q|^2 \dd |q| \dd \Omega_q}{2q^0} \delta(s - 2\sqrt{s}q^0) \\
	 &= \frac{\lambda^2}{2} \frac{i}{8\pi^2} \int \sqrt{(q^0)^0 - M^2} \dd q^0 \dd \Omega_q \delta(s - 2\sqrt{s}q^0) \\
	 &= \frac{\lambda^2}{2} \frac{i}{8\pi^2} \frac{\sqrt{s/4 - M^2}}{2\sqrt{s}} \int \dd \Omega_q \\
	 &= \frac{\lambda^2}{2} \frac{i}{8\pi} \sqrt{1 - \frac{4M^2}{s}} \\
	\text{Im}\M &= \frac{\lambda^2}{4} \frac{1}{8\pi} \sqrt{1 - \frac{4M^2}{s}}
\end{align*}
Note $\sigma = \frac{\lambda^2}{32\pi}$ and $2F = s \sqrt{1-\frac{4M^2}{s}}$. Thus optical theorem is still valid.

We can do more. Construct the complete $\M(s)$ from $\Im\M(s)$ through a \underline{dispersion relation}!

\begin{center}
\tikzset{zigzag/.style={decorate, decoration=zigzag}}
\usetikzlibrary{decorations.markings}
\tikzset{decoration={
    markings,
    mark=at position 0.5 with {\arrow{>}}}}
\begin{tikzpicture}[scale=1, transform shape]
	\draw [thick, <->] (0,3) -- (0,0) -- (3,0);
	\draw [thick, -] (0,-3) -- (0,0) -- (-3,0);
	\node [above] at (0,3) {$\Im(s)$};
	\node [right] at (3,0) {$\Re(s)$};
	\draw [fill] (1,0) circle [radius=0.05];
	\node [below] at (1,0) {$4M^2$};
	\draw [thick, zigzag, red] (1,0) -- (3,0);

	\draw [fill] (1.5,1) circle [radius=0.05];
	\draw (1.5,1) circle [radius=0.5];
	\node [below] at (1.5,1) {$s$};
	\draw[postaction={decorate}] (2.5,0.2) to [out=90, in=0] (0,2);
	\draw[postaction={decorate}] (0,2) to [out=-180, in=90] (-2.5,0);
	\draw[postaction={decorate}] (0,-2) to [out=0, in=-90] (2.5,-0.2);
	\draw[postaction={decorate}] (-2.5,0) to [out=-90, in=180] (0,-2) ;

	\draw (0.8,0.2) -- (2.5,0.2);
	\draw (0.8,-0.2) -- (2.5,-0.2);
	\draw (0.8, 0.2) -- (0.8, -0.2);

	\node at (2.8,0.5) {$C^+$};
	\node at (2.8,-0.5) {$C^-$};
\end{tikzpicture}
\end{center}

Use Cauchy's theorem:
\begin{align}
	\M(s) &= \frac{1}{2\pi i} \oint \frac{\M(z)\dd z}{z-s} \\
	\shortintertext{dropping the large circle}
		  &\longmapsto  \frac{1}{2\pi i }\int_{C_+ + C_-} \frac{\M(z)\dd z}{z-s}\notag\\
	&= \frac{1}{2 \pi i} \left[ \int^\infty_{4M^2}\frac{M(z+i\epsilon)\dd z}{z-s} - \int^\infty_{4M^2}\frac{M(z-i\epsilon)\dd z}{z-s} \right] \notag\\
	&= \frac{1}{2\pi i }\int^\infty_{4M^2} \frac{\text{disc}\M(z)\dd z}{z -s } \notag\\
	&= \frac{1}{\pi} \int^\infty_{4M^2} \frac{\Im\M(z) \dd z}{z-s} 
\end{align}

Repeat the exercise for $\frac{\M(s)-\M(0)}{s}$ (no pole introduced!).
\begin{align*}
	\Im \left( \frac{\M(s)-\M(0)}{s} \right) &= \frac{\Im\M(s)}{s} \\
	\M(s) - \M(0) &= \frac{s}{\pi} \int^\infty_{4M^2} \frac{\Im\M(z)\dd z}{z(z-s)} \\
				  &= \frac{\lambda^2}{2} \frac{s}{(4\pi)^2} \int^\infty_{4M^2} \frac{\dd z}{z(z-s)} \sqrt{1-\frac{4M^2}{z}} \\
	\shortintertext{using $\sigma = \sqrt{1-\frac{4M^2}{s}}$ and $\zeta = \sqrt{1 - \frac{4M^2}{z}}$}
				&= \frac{\lambda^2}{2}\frac{1}{8\pi^2} \int^1_0 \frac{\zeta^2}{\zeta^2 - \sigma^2} \dd \zeta \\
				&= \frac{\lambda^2}{2}
	\begin{cases}
		\frac{1}{8\pi^2} \left( 1 -\frac{\sigma}{2}\log{\frac{\sigma+1}{\sigma-1}} \right) & s < 0 \Leftrightarrow \sigma > 1 \\
		\frac{1}{8\pi^2} \left( 1- \sqrt{-\sigma^2} \arctan{\frac{1}{\sqrt{-\sigma^2}}} \right) & 0 < s < 4M^2, \sigma^2 < 0 \\
		\frac{1}{8\pi^2} \left( 1 - \frac{\sigma}{2}\log{\frac{1+\sigma}{1-\sigma}} + \frac{\textcolor{red}{i}\sigma}{16\pi} \right) & s > M^2, 0 < \sigma < 1
	\end{cases}
\end{align*}

Note: we are going to calculate this diagram again, noticing that $\int \frac{\dd^4 q}{(q^2 \dots)(q^2 \dots)}$ is logarithmically divergent!. The above representation demonstrates that this divergence resides in $M(0)$!

\section{Field-strength renomrlization}
What is structure of the propagator $\braket{\Omega | T\phi(x)\phi(y) | \Omega}$ at higher orders? At lower order
\begin{align*}
	\feynmandiagram[small, horizontal=a to b]{a --[fermion, edge label=\(p\)]b;};
	= \frac{i}{p^2 - M^2 + i\epsilon}
\end{align*}
Beyond this the propagator is not a simple pole. In $\phi^3$-theory 
\feynmandiagram[layered layout, small, horizontal=a to b, baseline=(a.base)]{
	a -- x;
	x --[half left] y;
	x --[half right] y;
	y -- b;
};
branch cuts are at $p^2 \leq 4M^2$.
In $\phi^4$-theory
\feynmandiagram[layered layout, small, horizontal=a to b, baseline=(a.base)]{
	a -- x;
	x --[half left] y;
	x --[half right] y;
	x -- y;
	y -- b;
};	
branch cuts are at $p^2 \leq 9M^2$. To induce cuts in the analytic structure.

Insert complete set of intermediate states ($x^0 > y^0$)
\begin{align*}
	\braket{\Omega | T\phi(x)\phi(y) | \Omega} = \sum_\lambda \int \frac{\dd^3 p}{(2\pi)^3 2E_p(\lambda)} \braket{\Omega | \phi(x) | \lambda_{\pmb{p}}} \braket{\lambda_{\pmb{p}} | \phi(y) | \Omega}
\end{align*}
with
\begin{itemize}[label={}]
	\item $\lambda$ multiparticle state
	\item $\lambda_0$ "rest frame", i.e.~$\hat{\pmb{P}}\ket{\lambda_0} = 0$
	\item $\lambda_{\pmb{p}}$ boosted to momentum $\pmb{p}$
\end{itemize}

Call energy of $\lambda_0 = m_\lambda$. From single particle to multi particle $E_{\pmb{p}}(\lambda) = \sqrt{m^2_\lambda + |\pmb{p}|^2}$.

\begin{align*}
	\braket{\Omega | \phi(x) | \lambda_{\pmb{p}}} &= \braket{\Omega| e^{i\hat{P}x} \phi(0) e^{-i\hat{P}x}|\lambda_{\pmb{p}}}	\\
									   &= \left. \braket{\Omega| \phi(0) | \lambda_{\pmb{p}}} e^{-ipx} \right\rvert_{p^0 = E_{\pmb{p}}} \\
									   \shortintertext{$\Omega$ and $\phi(0)$ are invariant under momentum boost}
									   &= \left. \braket{\Omega| \phi(0) | \lambda_{0}} e^{-ipx} \right\rvert_{p^0 = E_{\pmb{p}}} \\
\end{align*}

\begin{align}
	\braket{\Omega | T\phi(x)\phi(y) | \Omega} &= \sum_\lambda \int \frac{\dd^3 p}{(2\pi)^3 2E_p(\lambda)} e^{-ip(x-y)} |\braket{\Omega| \phi(0) | \lambda_0 }|^2 \\
											   &= \sum_\lambda \underbrace{\int \frac{\dd^4 p}{(2\pi)^4} \frac{i}{p^2 - m^2_\lambda + i\epsilon} e^{-ip(x-y)}}_{D_F(x-y;m^2_\lambda) \text{ when combined with } y^0 > x^0} |\braket{\Omega| \phi(0) | \lambda_0 }|^2 \\ 
\end{align}

Formally write this as 
\begin{align}
	\braket{\Omega | T\phi(x)\phi(y) | \Omega} = \int^\infty_0 \frac{\dd s}{2\pi} \rho(s) D_F(x-y;s)
\end{align}
with $\rho(s)$ the spectral density function.
\begin{align}
	\rho(s) \defeq \sum_\lambda (2\pi) \delta(s - m_\lambda^2) \left| \braket{\Omega | \phi(0) | \lambda_0} \right|^2
\end{align}

A typical spectral function looks like
\begin{figure}[htpb]
\begin{center}
\begin{tikzpicture}[scale=1, transform shape]
	\draw [fill] (1,0) circle [radius=0.05];
	\node [below] at (1,0) {$m^2$};
	\draw (1,3.5) -- (1,0);
	\node [above, align=center] at (1,3.7) {\small single\\ \small particle};

	\draw (3,2.5) -- (3,0);
	\draw (2.5,3) -- (2.5,0);
	\node [above, align=center] at (2.75, 2.9) {\small potential 2-particle \\ \small bound states};

	\draw [draw opacity=0, fill=lightgray] (4,0) to [out=90, in=180] (4.8,2) to [out=0, in=180] (7.5,1) -- (7.5,0) ;
	\draw (4,0) to [out=90, in=180] (4.8,2) to [out=0, in=180] (7.5,1);
	\draw [fill] (4,0) circle [radius=0.05];
	\node [below] at (4,0) {$4m^2$};
	\node [above] at (6, 2) {multi-particle continuum};

	\draw [thick, <->] (0,5) -- (0,0) -- (8,0);	
	\node [above] at (0,5) {$\rho(s)$};
	\node [below] at (8,0) {$s$};
\end{tikzpicture}
\end{center}
\caption{typical spectral function}
\label{fig:specFunc}
\end{figure}

Single particle contribution
\begin{align}
	\rho(s) = 2\pi \delta(s-m^2)Z + (\text{contributions} \geq 4m^2)
\end{align}
with $Z = \left| \braket{\Omega | \phi(0) | \lambda_{0}}\right|^2$ the field-strength renomrlization factor. 

Fourier transforming two-point function
\begin{align*}
	&\int \dd^4 x e^{ipx} \braket{\Omega | T\phi(x)\phi(0) | \Omega} \\
	=& \int^\infty_0 \frac{\dd s}{2\pi} \rho(s) \frac{i}{p^2 - s + i\epsilon} \\
	=& \frac{iZ}{p^2 - m^2 _ i\epsilon} + \int^\infty_{\sim 4m^2} \frac{\dd s}{2\pi} \rho(s) \frac{i}{p^2 - s + i\epsilon}
\end{align*}

Comparing to free theory: $\braket{0| \phi(0) | \pmb{p}}=1$ hence $Z=1$.

\section{LSZ reduction formula}

\section{The propagator(again)}
See also Peskin \& S. Chapter 10.2.

How do we calculate the propagtor and the wave-function renormalization factor $Z$ in perturbation theory, using Feynman diagrams? Call mass \underline{parameter} in $\lag = \frac{1}{2}(\partial_\mu \phi_0)^2 - \frac{m^2_0}{2}\phi_0^2 - \frac{\lambda_0}{4!}\phi_0^4$ $m_0$ \textit{bare mass}.

In $\phi^4$-theory "1-particle-irreducible" (1PI) contribution is
\begin{align*}
	-i\Sigma(p^2) = &
	\begin{tikzpicture}[scale=1, transform shape]
		\draw (0,0) -- (2,0);
		\draw (1,0.25) circle[radius=0.25];
	\end{tikzpicture} 
	+
	\begin{tikzpicture}[scale=1, transform shape]
		\draw (0,0) -- (2,0);
		\draw (1,0.25) circle[radius=0.25];
		\draw (1,0.75) circle[radius=0.25];
	\end{tikzpicture}
	+ 
	\begin{tikzpicture}[scale=1, transform shape,baseline=(a.base)]
		\draw (a) (0,0) -- (2,0);
		\draw (1,0) circle[radius=0.5];
	\end{tikzpicture}
	+ \dots \\
	\shortintertext{Then the \underline{complete} propagator using $D^0_F(p^2) = \frac{i}{p^2 - m^2_0 +i\epsilon}$ is} 
	D_F(p^2) =& 
	\begin{tikzpicture}[scale=1, transform shape]
		\draw (0,0) -- (2,0);
	\end{tikzpicture} 
	+
	\begin{tikzpicture}[scale=1, transform shape,baseline=(a.base)]
		\draw (a) (0,0) -- (0.6,0);
		\draw (a) (1.4,0) -- (2,0);
		\draw (1,0) circle[radius=0.4];
		\node at (1,0) {$-i\Sigma$};
	\end{tikzpicture}
	+
	\begin{tikzpicture}[scale=1, transform shape,baseline=(a.base)]
		\draw (a) (-0.2,0) -- (0.2,0);
		\draw (0.6,0) circle[radius=0.4];
		\node at (0.6,0) {$-i\Sigma$};
		\draw (1,0) -- (1.2,0);
		\draw (1.6,0) circle[radius=0.4];
		\node at (1.6,0) {$-i\Sigma$};
		\draw (2,0) -- (2.4,0);
	\end{tikzpicture} \\
	=& D^0_F(p^2) + D_F^0 (p^2) \left( -i\Sigma(p^2) \right)D^0_F(p^2)  +  D_F^0 (p^2) \left( -i\Sigma(p^2) \right) D_F^0 (p^2) \left( -i\Sigma(p^2) \right) D^0_F(p^2) \\
	\shortintertext{It is cleary a geometric series}
	=& \frac{D^0_F(p^2)}{1+i\Sigma(p^2)D_F^0(p^2)} = \frac{i}{p^2 - m_0^2 -\Sigma(p^2)}
\end{align*}
The pole of propagator does not occur at $m_0^2$ anynore. It will be shifted by $\Sigma \sim \mathcal{O}(\lambda)$!

Choose $m^2$ by the condition 
\begin{align}
	m_0^2 + \Sigma(m^2) = m^2
\end{align}
Expand
\begin{align}\label{math:Sigma}
	\Sigma(p^2) = \Sigma(m^2) + (p^2 - m^2) \Sigma'(m^2) + (p^2 - m^2)\tilde{\Sigma}(p^2)
\end{align}
where $\tilde{\Sigma}$ represents a correction (to first order Taylor expansion) and it satisfies $\tilde{\Sigma}(m^2) = 0$.

Then the propagator
\begin{align}
	D_F(p^2) =& \frac{i}{p^2 - m^2_0 - \Sigma(p^2)} = \frac{i}{(p^2 - m^2)(1+\frac{\Sigma(m^2)-\Sigma(p^2)}{p^2 - m^2})} \notag \\
	\shortintertext{using \ref{math:Sigma}}
	=& \frac{i}{(p^2 - m^2)(1-\Sigma'(m^2)-\tilde{\Sigma}(p^2))} \notag\\
	=& \frac{iZ}{p^2 - m^2} \cdot \frac{1}{1-Z\tilde{\Sigma}(p^2)} \notag \\
	=& \frac{iZ}{p^2 - m^2} + \text{regular}
\end{align}
with $Z = \left( 1 - \frac{\partial}{\partial p^2} \left.\Sigma(p^2)\right\rvert_{p^2 = m^2} \right)^{-1}$

Starting point Lagrangian is $\lag = \frac{1}{2}(\partial_\mu \phi_0)^2 - \frac{m^2_0}{2}\phi_0^2 - \frac{\lambda_0}{4!}\phi_0^4$. To remove $Z$ from numerator frome the propagator and instead put $\sqrt{Z}$ onto the couplings at each end. Since each internal vertex has 4 lines (remember the vertex carries the coupling constant)
\begin{align}
	\lambda_0 \longmapsto \lambda_1 = Z^2 \lambda_0
\end{align}

In $\Sigma$ and $\tilde{\Sigma}$, there are 2 external lines without $\sqrt{Z}$, so
\begin{align}
	\Sigma(p^2, \lambda_0, \text{old } D_F) = \frac{1}{Z} \Sigma_1(p^2, \lambda_1, \text{new } D'_F)
\end{align}
(same expression for $\tilde{\Sigma}$).

Thus we get the new propagator
\begin{align}
	D'_F(p^2) = \frac{i}{p^2 - m^2} \cdot \frac{1}{1-\tilde{\Sigma}_1(p^2)}
\end{align}
where $\tilde{\Sigma}_1(m^2) = 0$.

Define the renomalized field
\begin{align}
	Z^{-\frac{1}{2}} \phi_0 = \phi
\end{align}
Then $D'_F$ is the Fourier transform of $\braket{0 | T\phi(x)\phi(y) | 0}$

Rewrite the Lagrangian as
\begin{align}
	\lag = \frac{1}{2} \left( (\partial_\mu \phi)^2 - m^2 \phi^2 \right) \boxed{-\frac{\lambda_1}{4!}\phi^4 \underbrace{- \frac{1}{2}\delta m^2 \phi^2 + \frac{1}{2}(Z-1) \left( (\partial_\mu \phi)^2 - m^2 \phi^2 \right)}_{\text{counter-terms}} }
\end{align}
where $\delta m^2 = -Z(m^2 + m_0^2) = -Z\Sigma(m^2) = -\Sigma_1 (m^2)$. Everythin inside the $\boxed{\text{box}}$ can be considered as "interaction". May look weird given the kinetic/mass-like terms, but no contradiction. Consider just $\lag = \frac{1}{2}(\partial \phi)^2 - \frac{m^2}{2}\phi^2$. The mass-term $\equiv$ "interaction".

A massless propagator $\feynmandiagram[small, horizontal=a to b, baseline=(a.base)]{a--b}; = \frac{i}{p^2}$ and interaction $\feynmandiagram[small, horizontal=a to b, baseline=(a.base)]{a --[insertion=0.5] b;};=-im^2$. The resummed propagator is then
\begin{align*}
	\feynmandiagram[layered layout, horizontal=a to x, small, baseline=(x.base)]{
		a -- x[blob] -- b;
	};
	=&
	\feynmandiagram[small, horizontal=a to b, baseline=(a.base)]{a --b;};
	+
	\feynmandiagram[small, horizontal=a to b, baseline=(a.base)]{a --[insertion=0.5] b;};
	+
	\feynmandiagram[small, horizontal=a to b, baseline=(a.base)]{a --[insertion=0.33, insertion=0.66] b;};
	+ \dots \\
	=& \frac{i}{p^2} \left( 1 + \frac{i}{p^2}\cdot(-im^2)+\dots \right) \\
	=& \frac{i}{p^2} \left( 1 - (-im^2)\frac{i}{p^2} \right)^{-1} = \frac{i}{p^2 - m^2} 
\end{align*}

Actually this is not all. We will also have to further renomalize $\lambda_1$
\begin{align*}
	\begin{tikzpicture}[scale=1, transform shape, baseline=(x.base)]
		\begin{feynman}
			\vertex (x1);
			\vertex (x2) at (2,0);
			\vertex (x3) at (0,-2);
			\vertex (x4) at (2,-2);
			\vertex (x) at (1,-1);
			\diagram*{
				(x1)-- (x) --(x2),
				(x3)-- (x) --(x4),
			};
		\end{feynman}
	\end{tikzpicture}
	+
		\begin{tikzpicture}[scale=1, transform shape, baseline=(x.base)]
		\begin{feynman}
			\vertex (x1);
			\vertex (x2) at (2,0);
			\vertex (x3) at (0,-2);
			\vertex (x4) at (2,-2);
			\vertex (xx) at (1,-0.6);
			\vertex (y) at (1,-1.4);
			\diagram*{
				(x1)-- (xx) --(x2),
				(x3)-- (y) --(x4),
				(xx) --[quarter left] (y) --[quarter left] (xx),
			};
		\end{feynman}
	\end{tikzpicture}
	+
	\begin{tikzpicture}[scale=1, transform shape, baseline=(x.base)]
		\begin{feynman}
			\vertex (x1);
			\vertex (x2) at (2,0);
			\vertex (x3) at (0,-2);
			\vertex (x4) at (2,-2);
			\vertex (x) at (0.6,-1);
			\vertex (y) at (1.4,-1);
			\diagram*{
				(x1)-- (x) --(x3),
				(x2)-- (y) --(x4),
				(x) --[quarter left] (y) --[quarter left] (x),
			};
		\end{feynman}
	\end{tikzpicture}
	+
	\begin{tikzpicture}[scale=1, transform shape, baseline=(x.base)]
		\draw (0,0) to  (0.6,-0.5) to[out=-45, in=45] (0.6, -1.5) to (0,-2);
		\draw (0.6, -0.5) to [out=180, in=-180] (0.6,-1.5);
	\end{tikzpicture}
\end{align*}

%%%%%%%%%%%%%%%%%%% TODO: complete the rest

%%%%%%%%%%%%%%%%%%%%%%%%%%%%%%%%%%%%%%%%%%%
\begin{align*}
	\begin{tikzpicture}[scale=1, transform shape, baseline=(x.base)]
		\begin{feynman}
			\vertex (x1);
			\vertex (x2) at (2,0);
			\vertex (x3) at (0,-2);
			\vertex (x4) at (2,-2);
			\vertex (x) at (0.6,-1);
			\vertex (y) at (1.4,-1);
			\diagram*{
				(x1)-- (x) --(x3),
				(x2)-- (y) --(x4),
				(x) --[quarter left] (y) --[quarter left] (x),
			};
		\end{feynman}
	\end{tikzpicture}
	= \frac{\mu^{2(4-d)\lambda^2}\lambda^2}{2} \int^1_0 \dd x \int \frac{\dd^d k}{(2\pi)^d} \frac{1}{[k^2 - \Delta(x)]^2} \\
	= \frac{\lambda^2}{2} \mu^{2(4-d)} \int_0^1 \dd x \frac{1}{(4\pi)^{d/2}} \frac{\Gamma(2-d/2)}{\Gamma(2)} \frac{1}{\Delta(x)^(2-d/2)} \\
	= \frac{\lambda^2}{2} \frac{\mu^{4-d}}{(4\pi)^2} \left\{ -2 \left[ \frac{1}{d-4} + \frac{1}{2}(\gamma_E - \log{4\pi} + \log(\frac{M}{\mu})) \right] - \int^1_0 \dd x \log{\frac{\Delta(x)}{M^2}} \right\} \quad \Delta(x)= M^2 - x(1-x)p^2 \\ 
	\int_0^1 \dd x \log{\frac{M^2-x(1-x)p^2}{M^2}} = \int_0^1 \dd x \log{ \left[ (\frac{\sigma+1}{2} - x)(x+\frac{\sigma-1}{2}) \right] } - \log{\frac{\sigma^2 - 1}{4}} , \quad \sigma = \sqrt{1-\frac{4M^2}{p^2}} \\
	= \sigma \log{\frac{\sigma+1}{\sigma-1}} - 2
\end{align*}
Valid for $p^2 < 0$, rest by analytic contination

Compare $M(s) - M(0)$ calculated based on Cutkosky and dispersion integral. Easier 
\begin{align*}
	M(0) = \frac{1}{i} \int \frac{\dd^d k}{(2\pi)^d} \frac{1}{(k^2 - M^2)^2} = \frac{\partial}{\partial M^2} \frac{1}{i} \int \frac{\dd^d k}{(2\pi)^d} \frac{1}{k^2 - M^2} \\
	= \frac{\partial}{\partial M^2} \left\{ -\frac{M^2}{8\pi^2} \left[ \frac{1}{d-4} + \frac{1}{2}(\gamma_E - 1 -\log{4\pi}) + \frac{1}{2} \log{\frac{M^2}{\mu^2}} \right] \right\} \\
	\shortintertext{$1$ cancalled by the derivative of $\log$}
	= - \frac{1}{8\pi^2} \left[ \frac{1}{d-4} + \frac{1}{2} (\gamma_E - \log(4\pi) + \frac{1}{2} \log{\frac{M^2}{\mu^2}}) \right]
\end{align*}

Lets summarise the renormalization of $\phi^4$ at one loop
\begin{itemize}
	\item   %TODO diagram
		is \underline{indeoendt} of $p^2$! Hence $\Sigma(p^2)$ at $\mathcal{O}(\lambda)$ only renormalises the \underline{mass}, there is no wavefunction renormalisation $Z (\sim \frac{\partial \Sigma}{\partial p^3} |_{p^2 = M^2})$. Thus $Z=1 + \mathcal{O}(\lambda^2)$
	This does change at $\mathcal{O}{\lambda^2}$: % TODO: diagram 
	$\rightarrow Z\neq 1$
\item Mass renomalisation
	% TODO: diagram
	Then 
	\begin{align*}
	M^2 = M^2 + \frac{\lambda M^2}{16\pi^2} \left[ \frac{1}{d-4} + \frac{1}{2} (\gamma_E - 1 -\log{4\pi} + \log{\frac{M}{\mu}}) \right]- M^2 + M^2 \\
	= M^2_0 + \frac{\lambda M^2}{16\pi^2} \left[ \frac{1}{d-4}  + \frac{1}{2}(\gamma_E - 1 - \log{4\pi} + \log{\frac{M}{\mu}}) + \mathcal{O}(\lambda, (d-4)) \right] \\
	M^2_\text{physical} \neq f(\mu), \quad \lambda \mu^{4-d} M^2 = \lambda_0 M_0^2 + \mathcal{O}(\lambda^2) \text{ and $\lambda_0$ and $M_0$ are independent of $\mu$}
	\end{align*}
	\item Coupling constant renormlaisation. Lets choose renormlisation point for $\lambda$ at $s=t=u=0$ for simplicity:
		%TODO: diagram
		with $Z=1$
		\begin{align*}
			\lambda_0 =  \lambda \mu^{4-d}Z_\lambda = \lambda \mu^{4-d} \left\{ \underbrace{1- \frac{3}{\lambda}{16\pi^2} [ \frac{1}{d-4}}_{Z^MS_\lambda \text{ minimal subtraction}} +\frac{1}{2} (\gamma_E -\log{4\pi} + \log{\frac{M}{\mu}})] + \mathcal{O})\lambda^2 \right\} \\
			\lambda_0 =  \lambda \mu^{4-d}Z_\lambda = \lambda \mu^{4-d} \left\{ \underbrace{1- \frac{3}{\lambda}{16\pi^2} [ \frac{1}{d-4} +\frac{1}{2} (\gamma_E -\log{4\pi} }_{Z^{\bar{MS}}_\lambda \text{modified minimal subtraction}} + \log{\frac{M}{\mu}})] + \mathcal{O})\lambda^2 \right\} \\
			\shortintertext{\text{these two $Z$ are mass-indepent}}
			\lambda_0 =  \lambda \mu^{4-d}Z_\lambda = \lambda \mu^{4-d} \left\{ \underbrace{1- \frac{3}{\lambda}{16\pi^2} [ \frac{1}{d-4} +\frac{1}{2} (\gamma_E -\log{4\pi}  + \log{\frac{M}{\mu}})] }_{Z_\lambda \text{mass-dependent}}+ \mathcal{O})\lambda^2 \right\} 
		\end{align*}
\end{itemize}

\section{Superficial defree of divergence}
How do we know that we are done renormalising the theory with
\begin{itemize}
	\item wave function
	\item mass
	\item coupling
\end{itemize}
Can't there be more divergences?

Want to analyse superficial degree of divergence $D$ of an arbitary loop diagram with 
\begin{itemize}
	\item $d$ dimension
	\item $L$ number of loops
	\item $I$ number of internal propagators
	\item $E$ number of external lines
	\item $V$ number of vertices
\end{itemize}

Matrix element of an arbitary diagram generically 
\begin{align*}
	\sim \lambda^V \int \frac{\dd^d k_1 \dd^d k_2 \dots \dd^d k_L}{(k_{i_1}^2 - M^2) \dots (k_{i_I}^2 - M^2)}
\end{align*}
ro clearly 
\begin{align}
	D = d L - 2 I
\end{align}

$D \geq 0 $ divergend ($D=0$ logarithmically divergent) and $D < 0$ convergent.

Express $L$ and $I$ in terms of $V$ and $E$
\begin{itemize}
	\item 
		\begin{align}
			L = \text{nuber of undetermined intergration momenta} \notag \\
			= \text{number of propagators} - \text{number of momentum conservation at each vertex} + 1 \text{(because of overal momentum conservation)} \notag \\
			L = I - V + 1 \label{math:super1}
		\end{align}
	\item vertex linked to 4 legs, internal lines attached to 2 vertices, external line to 1
		\begin{align}
			4V = 2I + E \label{math:super2}
		\end{align}
\end{itemize}
solve \ref{math:super1} and \ref{math:super2} for $L$ and $I$
\begin{align}
	D = d + (d-4) V - (\frac{d}{2} - 1)E \\
	\shortintertext{in physical 4 dimension}
	D = 4 - E
\end{align}

\paragraph{Remarks}
\begin{itemize}
	\item for $d=4$, $D$ is \underline{independent} of $V$, only dependent on $E$.
	\item only a few small $E$ produce $D \geq 0$, here (in $\phi^4$)
		$E = 2 \feynmandiagram{a -- x[blob] -- b};$
		$E = 4 \feynmandiagram{a -- x[blob] -- b, c -- x -- d};$
	\item distringuish theories of different d
		\begin{itemize}
			\item $d < 4$: $D$ \underline{decreases} with $V$, only finite number of digrams (not n-point functions) diverges 
				\textbf{super-renormalisable}
			\item $d=4$: $D$ is independent of $V$, only a finite number of amplitudes diverges, but at each order in perturbation theory
				\textbf{renormalisable}
			\item $d>4$: $D$ frows with $V$, even ampitude becomes divergent at some prder in perturbation theory.
				\textbf{non-renormalisable}
		\end{itemize}
	\item alternative characterisation in terms of \underline{mass dimension} of coupling constant
		\begin{align*}
			\lag_{\phi^4} = -\mu^{4-d}\frac{\lambda}{4!} \phi^4 = - \frac{\tilde{\lambda}}{4!} \phi^4
		\end{align*}
		so $[\tilde{\lambda}] = 4-d$ in $d$ dimension; hence 
			\begin{itemize}
				\item $[\tilde{\lambda}] > 0$ super-renormalisable
				\item $[\tilde{\lambda}] = 0$ renormalisable
				\item $[\tilde{\lambda}] < 0$ non-renormalisable
			\end{itemize}
		\item why is this "superficial"? There can always be divergent subgraphs! These subgraphs are regularised and renormalised by the treatment of the "primitive divergences" we have already seen before.
\end{itemize}
\paragraph{conclusion for $\phi^4$}
the only primitive divergences are $E=2$ and $E=4$ (and $E=0$ the vacuum graphs) and we renormalise the theory by  
\begin{align}
	M_0^2 = M^2 \left\{ 1 + c_m^{(1)}\frac{\lambda}{d-4} + c_m^{(2)}\frac{\lambda^2}{(d-4)^2} + \dots \right\} \\
	\lambda_0 = \lambda \left\{ 1 + c_\lambda^{(1)}\frac{\lambda}{d-4} + c_\lambda^{(2)}\frac{\lambda^2}{(d-4)^2} + \dots \right\} \\
	Z = 1 + c_z^{(2)} \frac{\lambda^2}{(d-4)^2} + \dots
\end{align}


\end{document}
