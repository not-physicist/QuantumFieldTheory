\documentclass{scrartcl}
\KOMAoptions{fontsize=11pt, paper=a4}     
\KOMAoptions{DIV=13}                      

\usepackage[utf8]{inputenc}               
\usepackage[T1]{fontenc}                  
\usepackage[autostyle=true]{csquotes}     
\usepackage[varg]{txfonts}  			  %	Times-like fonts in support of mathematics
\usepackage{siunitx}   	  				  
\usepackage{enumitem}				      %	extra enumerate options

\renewcommand{\familydefault}{\rmdefault} % font to sans serif

%import external graphics and where to find these
\usepackage{graphicx}					  
\graphicspath{{figs/}}

\RequirePackage[backend=biber, style=numeric]{biblatex}
\addbibresource{refs.bib}

\usepackage{hyperref}
\RequirePackage[all]{hypcap}

%There are a number of symbols defined inside txfonts that are also defined in amsmath
% so you can just make these available again
\let\iint\relax
\let\iiint\relax
\let\iiiint\relax
\let\idotsint\relax
\usepackage{amsmath}
\usepackage{physics}
\usepackage{mathtools}
\usepackage{braket}
\usepackage{slashed} % feynman slash notation

% This will set fancy headings to the top of the page. The page number will be
% accompanied by the total number of pages. That way, you will know if any page
% is missing.
% If you do not want this for your document, you can just use
% ``\pagestyle{plain}``.
\usepackage{scrpage2}
\usepackage{lastpage}
\pagestyle{scrheadings}
\automark{section}
\cfoot{\footnotesize{Seite \thepage\ / \pageref*{LastPage}}}
\chead{}
\ihead{}
\ohead{\rightmark}
\setheadsepline{.4pt}

%symbols for footnotes
\usepackage[symbol]{footmisc}
\renewcommand{\thefootnote}{\fnsymbol{footnote}} % footnote mark with special symbols

% toprule and etc.
\usepackage{booktabs}

%%%%%%%%%%%%%%%%%%%%%%%%%% NEW COMMAND SECTION %%%%%%%%%%%%%%%%%%%%
% undertilde
\usepackage{accents}
\newcommand{\utilde}[1]{\underaccent{\tilde}{#1}}

%define equal
\newcommand{\defeq}{\vcentcolon =} 
\newcommand{\eqdef}{= \vcentcolon}
\newcommand{\euler}{\mathrm{e}}

%Lagrange density
\newcommand{\lag}{\mathcal{L}} 
%Hamiltonian density
\newcommand{\ham}{\mathcal{H}}

%identity matrix
\usepackage{dsfont}
\newcommand{\id}{\mathds{1}}

%%%%%%%%%%%%%%%%%%%%%%%%%%%%% SETTINGS %%%%%%%%%%%%%%%%%%%%%%%%%%%%%%%%%%%%

\numberwithin{equation}{section}

%%%%%%%%%%%%%%%%%%%%%%%%%%%%%%%%%%%%%%%%%%%%%%%%%%%%%%%%%%%%%%%%%%

\title{Formula for QFT I}
\begin{document}
\maketitle

\section{Classical field theory}
\subsection{Field theory in continuum}
\paragraph{Euler-Lagrange-equation}
\begin{align}
	\partial_\mu \left( \frac{\partial \lag}{\partial(\partial_\mu \phi)} \right) - \frac{\partial \mathcal{L}}{\partial \phi} = 0
\end{align}
\paragraph{momentum density}
\begin{align}
	\pi(x) = \frac{\partial \lag}{\partial \dot{\phi}(\pmb{x})}
\end{align}
\paragraph{Hamiltonian density}
\begin{align}
	\ham(\phi(\pmb{x}),\pi(\pmb{x})) = \pi(\pmb{x}) \dot{\phi}(\pmb{x}) - \lag(\phi,\partial_\mu \phi)
\end{align}
\subsection{Noether Theorem}
If a Lagrangian field theory has an infinitisimal symmetry, then there is an associated current $j^\mu$, which is conserved.
\begin{align}
	\partial_\mu j^\mu &= 0 \\
	j^\mu &= \frac{\partial \lag}{\partial (\partial_\mu \phi)} \Delta \phi - X^mu
\end{align}
\paragraph{Energy-momentum tensor (stress-energy tensor)} \hspace{0pt}\\
Asymmetric version
\begin{align}
	\Theta^\mu_\nu = \frac{\partial \lag}{\partial(\partial_\mu \phi)} \partial_\nu \phi - \delta^\mu_\nu \lag
\end{align}
General version
\begin{align}
	T^{\mu\nu} = \Theta^{\mu\nu} + \partial_\lambda f^{\mu\nu\lambda}
\end{align}
with $f^{\lambda\mu\nu}=-f^{\mu\lambda\nu}$ or $\partial_\mu \partial\nu f^{\lambda\mu\nu}=0$

\section{Klein-Gordon theory}
\paragraph{(Real) Lagrangian density}
\begin{align}
	\lag = \frac{1}{2} \partial_\mu \phi \partial^\mu \phi - \frac{m^2}{2} \phi^2
\end{align}
\paragraph{Quantization}
\begin{align}
	\begin{split}
		\left[ \phi(\pmb{x}), \phi(\pmb{x}') \right] = \left[ \pi(\pmb{x}), \pi(\pmb{x}') \right] = 0 \\
		\left[ \phi(\pmb{x}), \pi(\pmb{x}') \right] = i \delta^{(3)}(\pmb{x} - \pmb{x}')
	\end{split}
\end{align}
\paragraph{Decomposition into Fourier modes}
\begin{align}
	\phi({\pmb{x}}) &= \int \frac{\dd^3 p}{(2\pi)^3}\frac{1}{\sqrt{2E_p}} \left( a_{\pmb{p}} e^{i\pmb{p}\cdot\pmb{x}} + a_{\pmb{p}}^\dagger e^{-i \pmb{p} \cdot \pmb{x}} \right) \\
	\pi({\pmb{x}}) &= \int \frac{\dd^3 p}{(2\pi)^3}(-i)\sqrt{\frac{E_p}{2}} \left( a_{\pmb{p}} e^{i\pmb{p}\cdot\pmb{x}} - a_{\pmb{p}}^\dagger e^{-i \pmb{p} \cdot \pmb{x}} \right)
\end{align}
thus the commutation relations for ladder operators:
\begin{align}
	\left[ a_{\pmb{p}}, a_{\pmb{p}'} \right] &= \left[ a^\dagger_{\pmb{p}}, a^\dagger_{\pmb{p}'} \right] = 0 \\
	\left[ a_{\pmb{p}}, a^\dagger_{\pmb{p}'} \right] &= (2\pi)^3 \delta^{(3)}(\pmb{p}-\pmb{p}')
\end{align}
\paragraph{Hamiltonian in terms of ladder operator}
\begin{align}
	H = \int \frac{\dd^3 p}{(2\pi)^3} E_p \left( a_{\pmb{p}} a_{\pmb{p}} ^\dagger + \frac{1}{2} \left[ a_{\pmb{p}}, a_{\pmb{p}}^\dagger \right] \right)
\end{align}
\paragraph{Normlisation} it's also lorentz-invariante
\begin{align}
	\braket{p|q} = 2E_p (2\pi)^3 \delta^{(3)}(\pmb{p} - \pmb{q})
\end{align}
\section{Heisenberg-picture fields}
\paragraph{Heisenberg-picture}
\begin{align}
	\ket{\psi_H} &= e^{iHt}\ket{\psi_s (t)} \\
	O_H(t) & = e^{iHt}O_S e^{-iHt} 
\end{align}
\paragraph{Field operator}
\begin{align}
	\phi(x) = \phi({\pmb{x},t}) &= \int \frac{\dd^3 p}{(2\pi)^3}\frac{1}{\sqrt{2E_p}} \left( a_p e^{ipx} + a_{p}^\dagger e^{-i p x} \right)
\end{align}

\section{Commutations and propogators}
\paragraph{Commutations}
\begin{align}
	\left[ \phi(x), \phi(y) \right] &= D(x-y) - D(y-x)
	\begin{cases}
		= 0 & \text{if $(x-y)$ is space-like} \\
		\neq 0 & \text{otherwise}
	\end{cases} \\
	D(x-y) &= \int\frac{\dd^3p}{(2\pi)^3} \frac{1}{2E_p} e^{-ip(x-y)}
\end{align}
\paragraph{Propogator}
\begin{align}
	\braket{0 | \phi(x) \phi(y) | 0} = D(x-y)
\end{align}
\paragraph{Feynman propagator}
\begin{align}
	\begin{split}
	D_F(x-y) &= \braket{0 | T \phi(x) \phi(y) | 0} \\
			 &= \Theta(x^0 - y^0)D(x-y) + \Theta(y^0 - x^0) D(y-x)
	\end{split}\\
	D_F(x-y) &= \int \frac{\dd^4 p}{(2\pi)^4}\frac{i}{p^2-m^2+i\epsilon} e^{-ip(x-y)}
\end{align}

\section{Quantization of the Dirac field}
\subsection{Dirac equation}
\begin{align}
	\left(i \gamma^\mu \partial_\mu - m \right) \phi(x) = 0
\end{align}
\paragraph{Standard representation (Dirac's)}
\begin{align}
	\gamma_0 = \begin{pmatrix} \id_2 & 0 \\ 0 & -\id_2 \end{pmatrix}
	\quad
	\pmb{\gamma} = \begin{pmatrix} 0 & \pmb{\sigma} \\ -\pmb{\sigma} & 0 \end{pmatrix}
\end{align}
\paragraph{Lorentz transformation}
\begin{align}
	\Lambda = \exp(\frac{1}{2}\omega_{\mu\nu}M^{\mu\nu})
\end{align}
with $\omega$ set of parameters and $M$ the generator of Lie algebra.
\paragraph{Spinor representation}
\begin{align}
	S^{\rho\sigma} &= \frac{1}{4} \left[ \gamma^\rho, \gamma^\sigma \right] = \frac{1}{2i}\sigma^{\rho\sigma} \\
\end{align}
\paragraph{Spinor transformation}
\begin{align}
	S(\Lambda) &= \exp(\frac{1}{2} \omega_{\mu\nu}S^{\mu\nu}) \\
	\psi'_a(x) &= S_{ab}(\Lambda) \psi_b(\Lambda^{-1}x)
\end{align}
\paragraph{adjoint spinor}
\begin{align}
	\bar{\psi} = \psi^\dagger \gamma^0
\end{align}
\paragraph{Fifth gamma matrix}
\begin{align}
	\gamma^5 &\defeq i \gamma^0 \gamma^1 \gamma^2 \gamma^3 \\
	\left\{ \gamma^\mu, \gamma^5 \right\} &= 0 \\
	(\gamma^5)^2 &= \id_4
\end{align}
\paragraph{Plane wavesolutions}
\begin{align}
	\psi(x) = \begin{cases}
		u(p) e^{-ipx} & \text{positive frequency} \\
		v(p) e^{ipx} & \text{negative frequency}
	\end{cases}
\end{align}
\begin{align}
	u_s	(p) = \sqrt{E_p+m} \begin{pmatrix} \chi_s \\ \frac{\pmb{u}\cdot\pmb{p}}{E_p+m}\chi_s\end{pmatrix} e^{-ipx}
	v_s	(p) = \sqrt{E_p+m} \begin{pmatrix} \frac{\pmb{u}\cdot\pmb{p}}{E_p+m}\tilde{\chi}_s \\ \tilde{\chi}_s \end{pmatrix} e^{ipx}
\end{align}
with
\begin{align*}
	\chi_{\frac{1}{2}} = \begin{pmatrix} 1 \\ 0 \end{pmatrix} \quad x_{-\frac{1}{2}} = \begin{pmatrix} 0 \\ 1\end{pmatrix} \\
	\quad s=\pm\frac{1}{2} \quad \tilde{\chi}_s = \chi_{-s}
\end{align*}
\paragraph{Orthogonality of spinor}
\begin{align}
	\bar{u}_s(p) u_{s'}(p) &= -\bar{v}_s(p) v_{s'}(p) = 2m \delta_{ss'} \\
	\bar{u}_s(p) v_{s'}(p) &= 0
\end{align}
\paragraph{Spin sums}
\begin{align}
	\sum_{s} u_s(p)\bar{u}_s(p) &= \slashed{p} + m  \\
	\sum_{s} v_s(p)\bar{v}_s(p) &= \slashed{p} - m 
\end{align}

\subsection{Dirac Lagrangian and quantization}
\begin{align}
	\lag = \bar{\psi}(x) (i\slashed{\partial} - m) \psi(x)
\end{align}
\paragraph{Quantization}
\begin{align}
	\left\{ \psi_a(\pmb{x}), \psi_b^\dagger(\pmb{x}') \right\} = \delta_{ab}\delta^{(3)}(\pmb{x}-\pmb{x}') \\
	\left\{ \psi_a(\pmb{x}), \psi_b(\pmb{x}') \right\} = \left\{ \psi^\dagger_a(\pmb{x}), \psi^\dagger_b(\pmb{x}') \right\} = 0
\end{align}
\paragraph{Field operators}
\begin{align}
\psi(\pmb{x}) = \int \frac{\dd^3 p}{(2\pi)^3 \sqrt{2E_p}} \sum_s (a_{\pmb{p}}^s u_s(\pmb{p})e^{i\pmb{p}\cdot\pmb{x}} + b_{\pmb{p}}^{s\,\dagger} v_s(\pmb{p})e^{-i\pmb{p}\cdot\pmb{x}})
\end{align}
thus the anticommutations of ladder operators:
\begin{align*}
	\left\{ a_{\pmb{p}}^s, a_{\pmb{p'}}^{s'\,\dagger} \right\} &= \left\{ b_{\pmb{p}}^s, b_{\pmb{p'}}^{s'\,\dagger} \right\}= (2\pi)^3\delta_{ss'}\delta^{(3)}(\pmb{p}-\pmb{p}') \\
	\left\{ a, a \right\} &= \left\{ a^\dagger, a^\dagger \right\} = \dots = 0
\end{align*}
\paragraph{Hamiltonian in terms of Fourier modes (with normal ordering)}
\begin{align}
	H = \int \frac{\dd^3 p}{(2\pi)^3} \sum_s E_p (a_{\pmb{p}}^{s\,\dagger}a_{\pmb{p}}^{s}-b_{\pmb{p}}^{s\,\dagger}b_{\pmb{p}}^{s})
\end{align}
\subsection{Particles and antiparticles}
\begin{align}
	Q &= e \int \dd^3 x \psi^\dagger(x)\psi(x) \\
	:Q: &= e \int \frac{\dd^3 p}{(2\pi)^3} \sum_s (a_{\pmb{p}}^{s\,\dagger}a_{\pmb{p}}^s - b_{\pmb{p}}^{s\,\dagger}b_{\pmb{p}}^s)
\end{align}
\subsection{Dirac propagator and anticommutators}
\begin{align}
	\begin{split}
	S_{ab}(x-y) &= \left\{ \psi_a(x), \bar{\psi}_b(y) \right\} \\
				&= (i\slashed{\partial}+m) \left[ D(x-y)-D(y-x) \right]
	\end{split}
\end{align}
\paragraph{Time ordering of Dirac fields}
\begin{align}
	T(\phi_a(x)\bar{\psi}_b(y)) = \Theta(x^0 - y^0)\psi_a(x)\bar{\psi}_b(y) - \Theta(y^0 - x^0) \bar{\psi}_b(y)\psi_a(x)
\end{align}
\paragraph{Feynman propogator for the Dirac field}
\begin{align}
	S_F(x-y) = \braket{0| T \psi(x)\bar{\psi}(y) |0} = \int \frac{\dd^4 p}{(2\pi)^4} \frac{i(\slashed{p}+m)}{p^2-m^2+i\epsilon} e^{-ip\cdot(x-y)}
\end{align}

\subsection{Discrete symmetries of the Dirac Field}
\begin{center}
\begin{tabular}{c c c}
	\toprule
& orientation perserving & orientation not perserving \\
\midrule
	(ortho)chronous & $\mathcal{L}_+^{\uparrow}$ & $\mathcal{L}_-^\uparrow=\mathcal{P}\mathcal{L}_+^{\uparrow}$ \\
	non-orthochronous & $\mathcal{L}_-^\downarrow = \mathcal{T}\mathcal{L}_+^{\uparrow}$ & $\mathcal{L}_+^{\downarrow}=\mathcal{P}\mathcal{T}\mathcal{L}_+^{\uparrow}$ \\
	\bottomrule
\end{tabular}
\end{center}

\end{document}
